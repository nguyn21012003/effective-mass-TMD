%\documentclass{article}
\documentclass{article}
\usepackage[utf8]{vietnam}
\usepackage[utf8]{inputenc}
\usepackage{anyfontsize,fontsize}
\changefontsize[13pt]{13pt}	
\usepackage{commath}
\usepackage{parskip,setspace}
\usepackage[table]{xcolor}
\usepackage{amssymb}
\usepackage[d]{esvect}
\usepackage[nottoc,notlot,notlof]{tocbibind}
\usepackage{slashed,cancel}
\usepackage{indentfirst,titlesec}
\usepackage{pdfpages}
\usepackage{graphicx,subcaption,floatrow,adjustbox,rotating}
\usepackage{nccmath}
\usepackage{mathtools}
\usepackage{amsfonts,esint}
\usepackage[printonlyused,withpage]{acronym}
\usepackage{wrapfig}
\usepackage[toc,page]{appendix}
\usepackage{amsmath,systeme}
\usepackage[thinc]{esdiff}
\usepackage{hyperref,tikz}
\usepackage{bm,physics,nicematrix}
\usepackage[numbers,comma,sort&compress]{natbib}
%footnote
\usepackage{fancyhdr}
\usepackage{array,tocbibind}
\usepackage{enumitem}
\pagestyle{empty}


\usepackage{geometry}
\geometry{
	a4paper,
	total={170mm,257mm},
	left=20mm,
	top=20mm,
}



\newcommand{\image}[1]{
	\begin{center}
		\includegraphics[width=0.5\textwidth]{pic/#1}
	\end{center}
}




\renewcommand{\l}{\ell}
\newcommand{\dps}{\displaystyle}

\newcommand{\f}[2]{\dfrac{#1}{#2}}
\newcommand{\at}[2]{\bigg\rvert_{#1}^{#2} }


\renewcommand{\baselinestretch}{1.5}



\hypersetup{
	colorlinks=true,
	linkcolor=black,
	filecolor=magenta,      
	urlcolor=cyan,
	pdftitle={UT},
	citecolor=black,
	pdfpagemode=FullScreen,
}

\urlstyle{same}

\titleformat{\chapter}[display]
{\centering\large\bfseries} 
{\textbf{\MakeUppercase{\chaptername}} \ \thechapter\vspace{15pt}}{20pt}
{\large} 

%\newcommand{\thesistitlee}{Three-band tight binding model for TMD monolayers in the presence of a magnetic field}
\newcommand{\thesistitlee}{title}
\newcommand{\address}{NATIONAL UNIVERSITY OF HO CHI MINH CITY UNIVERSITY OF SCIENCE}
\newcommand{\department}{FACULTY OF PHYSICS - ENGINEERING PHYSICS}
\newcommand{\thesisauthor}{Tran Khoi Nguyen}
\newcommand{\thesisadvisor}{Your Advisor's Name}
\newcommand{\graddate}{Ho Chi Minh City, 2025}
\newcommand{\thesisdedication}{To all the Ph.D. pursuing brave souls}

\newlist{abbrv}{itemize}{1}
\setlist[abbrv,1]{label=,labelwidth=1in,align=parleft,itemsep=0.1\baselineskip,leftmargin=!}
\begin{document}
\setlength{\parindent}{20pt}
%\begin{titlepage}
%	\begin{center}
%		{\bfseries
%			
%			{\large {\bf \address}}\\
%			{{\textbf{\department}}}\\
%			{---------------------o0o--------------------}
%			\vspace{1.5cm}
%			
%			{\large {\bf UNDERGRADUATE THESIS}}\\
%			\vspace{3.0cm}
%			
%		}
%		
%	\end{center}
%	\textit{\textbf{\underline{Thesis title:}}}\\
%	\begin{center}
%		{\bfseries
%			
%			{\largerrr\thesistitlee}
%			\vspace{1in}
%			
%		}
%	\end{center}
%	\noindent
%	\makebox[\textwidth]{\hfill\makebox[3in]{\hrulefill}}\\
%	\begin{center}
%		\makebox[\textwidth]{\hfill\makebox[3in]{\hfill \textbf{Student: Tran Khoi Nguyen}\hfill}}
%		\makebox[\textwidth]{\hfill\makebox[3in]{\hfill \textbf{Supervisor: Dr. Huynh Thanh Duc}\hfill}}
%	\end{center}
%	\begin{tikzpicture}[remember picture, overlay]
%		\draw[line width=2pt]
%		([xshift=1.5cm, yshift=1.5cm] current page.south west)
%		rectangle
%		([xshift=-1.5cm, yshift=-1.5cm] current page.north east);
%	\end{tikzpicture}
%	\begin{center}
%		\vspace{2.5in}
%		{\graddate}
%	\end{center}
%\end{titlepage}
%\begin{titlepage}
%	\begin{center}
%		{\bfseries
%			
%			{\large {\bf \address}}\\
%			{{\textbf{\department}}}\\
%			\vspace{2.5cm}
%			
%			{\large {\bf UNDERGRADUATE THESIS}}\\
%			\vspace{3.0cm}
%			
%		}
%		
%	\end{center}
%	\textit{\textbf{\underline{Thesis title:}}}\\
%	\begin{center}
%		{\bfseries
%			
%			{\largerrr\thesistitlee}
%			\vspace{1in}
%			
%		}
%	\end{center}
%	\noindent
%	\makebox[\textwidth]{\hfill\makebox[3in]{\hrulefill}}\\
%	\begin{center}
%		\makebox[\textwidth]{\hfill\makebox[3in]{\hfill \textbf{Student: Tran Khoi Nguyen}\hfill}}
%		\makebox[\textwidth]{\hfill\makebox[3in]{\hfill \textbf{Supervisor: Dr. Huynh Thanh Duc}\hfill}}
%	\end{center}
%	\begin{center}
%		\vspace{2.5in}
%		{\graddate}
%	\end{center}
%\end{titlepage}
%
%\newpage
%\pagenumbering{roman}
%\pagestyle{fancy}
%\renewcommand{\headrulewidth}{0pt}
%\fancyhf{}
%\fancyfoot[C]{\hspace{0cm} \thepage}
%\setcounter{page}{1}
%\pagenumbering{arabic}
\section{Theory}
\noindent In the model introduced by Liu~\textit{et al.}, only the orbitals of the M atom are included. We denote the wave functions of the three orbitals of the M atom as
\begin{gather}
	\ket{\phi_{1}} = \ket{d_{z^{2}}} , \quad \ket{\phi_{2}} = \ket{d_{xy}} , \quad \ket{\phi_{3}} = \ket{d_{x^{2} - y^{2}}}.
\end{gather}
The Bloch wavefunction in this model has the form
\begin{gather}
	\psi_{\mathbf{k}}^{\lambda} (\mathbf{r}) = \sum_{j=1}^{3} C_{j}^{\lambda}(\mathbf{k}) \sum_{\mathbf{R}} e^{i \mathbf{k \cdot R}} \phi_{j}(\mathbf{r} - \mathbf{R}).
\end{gather}
The coefficents $C_{j}^{\lambda}(\mathbf{k})$ are the solutions of the eigenvalue equation
\begin{gather}
	\sum_{jj'}^{3} \left[H_{jj'}^{\text{TB}}(\mathbf{k}) - \varepsilon_{\lambda}(\mathbf{k}) S_{jj'}(\mathbf{k})\right] C_{j}^{\lambda}(\mathbf{k}) = 0,
\end{gather}
where
\begin{equation}
	\begin{aligned}
		H_{jj'}^{\text{TB}}(\mathbf{k}) = \sum_{\mathbf{R}} e^{i \mathbf{k \cdot R}} \bra{\phi_{j}(\mathbf{r})} H_{\text{1e}} \ket{\phi_{j'}(\mathbf{r - R})},
	\end{aligned}
\end{equation}
and
\begin{equation}
	\begin{aligned}
		S_{jj'}(\mathbf{k}) = \sum_{\mathbf{R}} \bra{\phi_{j}(\mathbf{r})} \ket{\phi_{j'}(\mathbf{r - R})} \approx \delta_{jj'}.
	\end{aligned}
\end{equation}

In the case $B \neq 0$, wave function can be expressed as the coefficients of $C_{ji}^{\lambda}$ in the tight-binding wave function
\begin{gather}
	\psi_{\lambda,\mathbf{k}}(\mathbf{r}) = \sum_{j}^{3}\sum_{i}^{2q} C_{ji}^{\lambda}(\mathbf{k}) \sum_{{\alpha}}^{N_{\text{UC}}} e^{i\mathbf{k}\cdot(\mathbf{R}_{\alpha} + \mathbf{r}_{i})} \phi_{j}(\mathbf{r} - \mathbf{R}_{\alpha} - \mathbf{r}_{i}).
\end{gather}
where $j=1,2,3$ and $i = 1 ... 2q$. We have shown that, under an uniform magnetic field, Bloch bands $\lambda$ construct Landau levels at small fields and become fractal-structured at strong fields, which is known as the Hofstadter butterfly.

We now consider a new basis consisting of three eigenfunctions of the angular momentum operators $L^2$ and $L_{z}$, for $l = 2, m = 0, \pm 2$,
\begin{gather}
	\ket{\tilde{\phi}_{1}} = \ket{d_{m = 0}}, \quad
	\ket{\tilde{\phi}_{2}} = \ket{d_{m = +2}}, \quad
	\ket{\tilde{\phi}_{3}} = \ket{d_{m = -2}}.
\end{gather}
The new basis can be obtained from the old one by the transformation
\begin{gather}
	\ket{\tilde{\phi}_{j}} = \sum_{j'} W_{j'j} \ket{\phi_{j}},
\end{gather}
where
\begin{gather}
	W =
	\begin{pNiceMatrix}
		1 & 0                   & 0                  \\
		0 & \frac{i}{\sqrt{2}}  & -\frac{i}{\sqrt{2}} \\
		0 & \frac{1}{\sqrt{2}} & \frac{1}{\sqrt{2}}
	\end{pNiceMatrix}.
\end{gather}
In particular, 
\begin{gather}
	\ket{\tilde{\phi}_{1}} = \ket{\phi_{1}},\\
	\ket{\tilde{\phi}_{2}} = \frac{i}{\sqrt{2}} \ket{\phi_{2}} + \frac{1}{\sqrt{2}} \ket{\phi_{3}},\\
	\ket{\tilde{\phi}_{3}} = -\frac{i}{\sqrt{2}} \ket{\phi_{2}} + \frac{1}{\sqrt{2}} \ket{\phi_{3}}.
\end{gather}
The TB Hamiltonian in new basis reads
\begin{equation}
	\begin{aligned}
		\tilde{H}^{\text{TB}} (\mathbf{k}) & = W^{\dagger} H^{\text{TB}}(\mathbf{k}) W,
	\end{aligned}
\end{equation}
where $H^{\text{TB}} = H^{\text{NN}}$ or $H^{\text{TNN}}$.

%\begin{figure*}[htb]
%	\centering
%	\includegraphics[width=0.75\linewidth]{imgs/BSDFT.pdf}
%	\caption{Adopted from Liu \textit{et al.}}
%\end{figure*}
%
%Ở {Hình~1}, tại điểm $K$, dải hoá trị có đóng góp chủ yếu từ hai orbital $d_{xy}$ và $d_{x^{2} - y^{2}}$, trong khi dải dẫn chủ yếu đến từ orbital $d_{z^{2}}$. Khi có mặt từ trường, vùng Brillouin zone bị thu nhỏ lại $2q$ lần so với vùng Brillouin gốc, khiến các điểm $K$ và $K'$ co lại và nằm gần điểm $\Gamma$. Do đó, trong bài này, chúng tôi chỉ xét phổ Hofstadter butterfly tại điểm $\Gamma$.
%
%Ở {Hình~2}, khi có từ trường, hàm sóng được biểu diễn tại $\Gamma = (0,0)$. Cụ thể, các hình {(a)}, {(b)}, {(c)} tương ứng với mức $2q$ (dải hoá trị), trong khi các hình {(d)}, {(e)}, {(f)} tương ứng với mức $2q+1$ (dải dẫn).

The cyclotron frequency can be obtained from the energy difference between two Landau levels
\begin{equation}
	\begin{aligned}
		\hbar \omega_{c}       & = E_{n+1} - E_{n}  ,
	\end{aligned}
\end{equation}
which gives
\begin{gather}
	\omega_{c}  = \frac{E_{n+1} - E_{n}}{\hbar}.
\end{gather}
On the other hand, the cyclotron frequency is also defined as
\begin{gather}
	\omega_{c}         = \frac{eB}{m^{*}}.
\end{gather}
Combining the two expressions, the effective mas can be written as 
\begin{gather}
	m^{*}  =  \frac{eB}{\omega_{c}} = \frac{eB}{\frac{E_{n+1} - E_{n}}{\hbar}} = \frac{eB \hbar}{E_{n+1} - E_{n}}.
\end{gather}


\newpage
\subsection{Effective mass}
\subsubsection*{Monolayer MoS$_{2}$}

\begin{figure}[htb]
	\begin{subfigure}{0.495\textwidth}
		\centering
		\includegraphics[width=\linewidth]{imgs/MoS2/ButAndWave_c.pdf}
	\end{subfigure}
	\begin{subfigure}{0.495\textwidth}
		\centering
		\includegraphics[width=\linewidth]{imgs/MoS2/ButAndWave_v.pdf}
	\end{subfigure}
	\caption{Hàm sóng của 2 mức Landau kế tiếp nhau. (a) là dải dẫn, (b) là dải hoá trị.}
\end{figure}

Cấu trúc dải năng lượng của MoS$_{2}$ khi chưa có từ trường cho thấy rằng ở dải hoá trị điểm $\Gamma$ có mức năng lượng là $E \approx -0.058$ (eV), vậy nên khi ở có sự xuất hiện của từ trường thì mức năng lượng của điểm $\Gamma$ này vẫn xuất hiện ở đây. 

Khối lượng hiệu dụng cho MoS$_{2}$ khi chưa có từ trường được tính bằng công thức $\frac{1}{m_{ij}*} =\frac{1}{\hbar^{2}} \frac{\partial^{2} E(\mathbf{k})}{\partial k_{i} k_{j}}$ cho biết rằng $m_{e} \approx 0.4178 m_{0}, m_{h} \approx 0.5325 m_{0}$ và $m_{r} \approx 0.2341$ cho trường hợp TNN và $m_{e} \approx 0.450 m_{0}, m_{h} \approx 0.6487 m_{0}$ và $m_{r} \approx 0.2659m_{0}$ cho trường hợp NN. Trong khi đó, ở bài của Goryca \textit{et al.}\cite{goryca2019}, họ đưa ra rằng $m_{r} \approx 0.27 \pm 0.01 m_{0}$.\\
Khi có từ trường lớn ví dụ như $B = 100 $T:
\begin{itemize}
	\item[a)] Nearest neighbor
	\begin{itemize}
		\item Ở thung lũng K, $m_{h} \approx 0.7011 m_{0},\; m_{e} \approx 0.4763 m_{0}$. 
		Trong đó $m_{r} \approx 0.2836 m_{0}$ tăng $\approx 6.7\%$.
		
		\item Ở thung lũng K', $m_{h} \approx 0.6597 m_{0},\; m_{e} \approx 0.4606 m_{0}$. 
		Trong đó $m_{r} \approx 0.2713 m_{0}$ tăng $\approx 2.0\%$.
	\end{itemize}
	\item[b)] Third nearest neighbor
	\begin{itemize}
		\item Ở thung lũng K, $m_{h} \approx 0.4205 m_{0},\; m_{e} \approx 0.3275 m_{0}$. 
		Trong đó $m_{r} \approx 0.1841 m_{0}$ tăng $\approx 10.17\%$.
		
		\item Ở thung lũng K', $m_{h} \approx 0.4043 m_{0},\; m_{e} \approx 0.3023 m_{0}$. 
		Trong đó $m_{r} \approx 0.1730 m_{0}$ tăng $\approx 3.53\%$.
	\end{itemize}
\end{itemize}

\begin{figure}[htb]
	\begin{subfigure}{0.495\textwidth}
		\centering
		\includegraphics[width=\linewidth]{imgs/MoS2/massK1.pdf}
	\end{subfigure}
	\begin{subfigure}{0.495\textwidth}
		\centering
		\includegraphics[width=\linewidth]{imgs/MoS2/massK2.pdf}
	\end{subfigure}
	\caption{Khối lượng hiệu dụng.}
\end{figure}
\newpage
\subsubsection*{Monolayer MoSe$_{2}$}
\begin{figure}[htb]
	\begin{subfigure}{0.495\textwidth}
		\centering
		\includegraphics[width=\linewidth]{imgs/MoSe2/ButAndWave_c.pdf}
	\end{subfigure}
	\begin{subfigure}{0.495\textwidth}
		\centering
		\includegraphics[width=\linewidth]{imgs/MoSe2/ButAndWave_v.pdf}
	\end{subfigure}
	\caption{Hàm sóng của 2 mức Landau kế tiếp nhau.}
\end{figure}
Cấu trúc dải năng lượng của MoSe$_{2}$ khi chưa có từ trường cho thấy rằng điểm $\Gamma$ không xuất hiện gần điểm K, vậy nên khi ở có sự xuất hiện của từ trường thì mức năng lượng của điểm $\Gamma$ này không xuất hiện ở đây. Bên cạnh đó, ba mức Landau đầu tiên sẽ là của điểm K', khác với WSe$_{2}$ là hai mức Landau đầu tiên là của điểm K'.

Khối lượng hiệu dụng cho MoSe$_{2}$ khi chưa có từ trường được tính bằng công thức $\frac{1}{m_{ij}*} =\frac{1}{\hbar^{2}} \frac{\partial^{2} E(\mathbf{k})}{\partial k_{i} k_{j}}$ cho biết rằng $m_{e} \approx 0.4770 m_{0}, m_{h} \approx 0.5887 m_{0}$ và $m_{r} \approx 0.2634 m_{0}$ cho trường hợp TNN và $m_{e} \approx 0.5226 m_{0}, m_{h} \approx 0.7512 m_{0}$ và $m_{r} \approx 0.3082m_{0}$ cho trường hợp NN.

Khi ở từ trường lớn ví dụ như $B = 100 $T
\begin{itemize}
	\item[a)] Nearest neighbor
	\begin{itemize}
		\item Ở thung lũng K, $m_{h} \approx 0.8100 m_{0},\; m_{e} \approx 0.5529 m_{0}$. Trong đó $m_{r} \approx 0.3286 m_{0}$ tăng $\approx 6.62\%$.
		
		\item Ở thung lũng K', $m_{h} \approx 0.7632 m_{0},\; m_{e} \approx 0.5331 m_{0}$. Trong đó $m_{r} \approx 0.3138 m_{0}$ tăng $\approx 1.82\%$.
	\end{itemize}
	\item[b)] Third nearest neighbor
	\begin{itemize}
		\item Ở thung lũng K, $m_{h} \approx 0.4205 m_{0},\; m_{e} \approx 0.3275 m_{0}$. 
		Trong đó $m_{r} \approx 0.1841 m_{0}$ tăng $\approx 10.17\%$.
		
		\item Ở thung lũng K', $m_{h} \approx 0.4043 m_{0},\; m_{e} \approx 0.3023 m_{0}$. 
		Trong đó $m_{r} \approx 0.1730 m_{0}$ tăng $\approx 3.53\%$.
	\end{itemize}
\end{itemize}

\begin{figure}[htb]
	\begin{subfigure}{0.495\textwidth}
		\centering
		\includegraphics[width=\linewidth]{imgs/MoSe2/massK1.pdf}
	\end{subfigure}
	\begin{subfigure}{0.495\textwidth}
		\centering
		\includegraphics[width=\linewidth]{imgs/MoSe2/massK2.pdf}
	\end{subfigure}
	\caption{Khối lượng hiệu dụng.}
\end{figure}

\newpage
\subsubsection*{Monolayer MoTe$_{2}$}
\begin{figure}[htb]
	\begin{subfigure}{0.495\textwidth}
		\centering
		\includegraphics[width=\linewidth]{imgs/MoTe2/ButAndWave_c.pdf}
	\end{subfigure}
	\begin{subfigure}{0.495\textwidth}
		\centering
		\includegraphics[width=\linewidth]{imgs/MoTe2/ButAndWave_v.pdf}
	\end{subfigure}
	\caption{Hàm sóng của 2 mức Landau kế tiếp nhau.}
\end{figure}
Cấu trúc dải năng lượng của MoTe$_{2}$ khi chưa có từ trường cho thấy rằng điểm $\Gamma$ có mức năng lượng là $E \approx -0.1075$ (eV), vậy nên khi ở có sự xuất hiện của từ trường thì mức năng lượng của điểm $\Gamma$ này vẫn xuất hiện ở đây. 

Khối lượng hiệu dụng cho MoTe$_{2}$ khi chưa có từ trường được tính bằng công thức $\frac{1}{m_{ij}*} =\frac{1}{\hbar^{2}} \frac{\partial^{2} E(\mathbf{k})}{\partial k_{i} k_{j}}$ cho biết rằng $m_{e} \approx 0.4318 m_{0}, m_{h} \approx 0.6044 m_{0}$ và $m_{r} \approx 0.2519 m_{0}$ cho trường hợp TNN và $m_{e} \approx 0.5913 m_{0}, m_{h} \approx 0.8975 m_{0}$ và $m_{r} \approx 0.3565 m_{0}$ cho trường hợp NN, đã được đưa ra trong công trình của Goryca \textit{et al.}\cite{goryca2019}. Khi chưa có từ trường, khối lượng hiệu dụng $m_{e}, m_{h}$ của MoTe$_{2}$ là lớn nhất trong 6 loại vật liệu. \\
Khi ở từ trường lớn ví dụ như $B = 90 $T
\begin{itemize}
	\item[a)] Nearest neighbor
	\begin{itemize}
		\item Ở thung lũng K, $m_{h} \approx 0.9774 m_{0},\; m_{e} \approx 0.6304 m_{0}$. 
		Trong đó $m_{r} \approx 0.3832 m_{0}$ tăng $\approx 7.49\%$.
		
		\item Ở thung lũng K', $m_{h} \approx 0.9142 m_{0},\; m_{e} \approx 0.6034 m_{0}$. 
		Trong đó $m_{r} \approx 0.3635 m_{0}$ tăng $\approx 1.96\%$.
	\end{itemize}
	\item[b)] Third nearest neighbor
	\begin{itemize}
		\item Ở thung lũng K, $m_{h} \approx 0.7704 m_{0},\; m_{e} \approx 0.4850 m_{0}$. 
		Trong đó $m_{r} \approx 0.2976 m_{0}$ tăng $\approx 18.14\%$.
		
		\item Ở thung lũng K', $m_{h} \approx 0.6322 m_{0},\; m_{e} \approx 0.4463 m_{0}$. 
		Trong đó $m_{r} \approx 0.2616 m_{0}$ tăng $\approx 3.85\%$.
	\end{itemize}
\end{itemize}
Trong bài nghiên cứu của Goryca \textit{et al.} \cite{goryca2019}, họ đưa ra rằng $m_{r} = 0.36 \pm 0.04 m_{0}$, tăng 25\% so với bài nghiên cứu của Korm\'{a}nyos \textit{et al.} \cite{kormanyos2015}. Trong bài này, với trường hợp TNN, khối lượng rút gọn $m_{r} = 0.2976 m_{0}$ tăng $\approx 18\%$ so với khi chưa có từ trường, với $m_{r} = 0.2519 m_{0}$
\begin{figure}[htb]
	\begin{subfigure}{0.495\textwidth}
		\centering
		\includegraphics[width=\linewidth]{imgs/MoTe2/massK1.pdf}
	\end{subfigure}
	\begin{subfigure}{0.495\textwidth}
		\centering
		\includegraphics[width=\linewidth]{imgs/MoTe2/massK2.pdf}
	\end{subfigure}
	\caption{Khối lượng hiệu dụng.}
\end{figure}

\newpage
\subsubsection*{Monolayer WS$_{2}$}
\begin{figure}[htb]
	\begin{subfigure}{0.495\textwidth}
		\centering
		\includegraphics[width=\linewidth]{imgs/WS2/ButAndWave_c.pdf}
	\end{subfigure}
	\begin{subfigure}{0.495\textwidth}
		\centering
		\includegraphics[width=\linewidth]{imgs/WS2/ButAndWave_v.pdf}
	\end{subfigure}
	\caption{Hàm sóng của 2 mức Landau kế tiếp nhau.}
\end{figure}

Cấu trúc dải năng lượng của WS$_{2}$ khi chưa có từ trường cho thấy rằng điểm $\Gamma$ có mức năng lượng là $E \approx -0.1075$ (eV), vậy nên khi ở có sự xuất hiện của từ trường thì mức năng lượng của điểm $\Gamma$ này vẫn xuất hiện ở đây. 

Khối lượng hiệu dụng cho WS$_{2}$ khi chưa có từ trường được tính bằng công thức $\frac{1}{m_{ij}*} =\frac{1}{\hbar^{2}} \frac{\partial^{2} E(\mathbf{k})}{\partial k_{i} k_{j}}$ cho biết rằng $m_{e} \approx 0.2956 m_{0}, m_{h} \approx 0.3845 m_{0}$ và $m_{r} \approx 0.1671 m_{0}$ cho trường hợp TNN và $m_{e} \approx 0.3195 m_{0}, m_{h} \approx 0.4348 m_{0}$ và $m_{r} \approx 0.1841 m_{0}$ cho trường hợp NN, đã được đưa ra trong công trình của Goryca \textit{et al.}\cite{goryca2019}. \\
Khi ở từ trường lớn ví dụ như $B = 100 $T
\begin{itemize}
	\item[a)] Nearest neighbor
	\begin{itemize}
		\item Ở thung lũng K, $m_{h} \approx 0.4735 m_{0},\; m_{e} \approx 0.3389 m_{0}$. 
		Trong đó $m_{r} \approx 0.1974 m_{0}$ tăng $\approx 7.2\%$.
		
		\item Ở thung lũng K', $m_{h} \approx 0.4438 m_{0},\; m_{e} \approx 0.3273 m_{0}$. 
		Trong đó $m_{r} \approx 0.1885 m_{0}$ tăng $\approx 2.3\%$.
	\end{itemize}
	\item[b)] Third nearest neighbor
	\begin{itemize}
		\item Ở thung lũng K, $m_{h} \approx 0.4205 m_{0},\; m_{e} \approx 0.3275 m_{0}$. 
		Trong đó $m_{r} \approx 0.1841 m_{0}$ tăng $\approx 10.17\%$.
		
		\item Ở thung lũng K', $m_{h} \approx 0.4043 m_{0},\; m_{e} \approx 0.3023 m_{0}$. 
		Trong đó $m_{r} \approx 0.1730 m_{0}$ tăng $\approx 3.53\%$.
	\end{itemize}
\end{itemize}
Như vậy ở thung lũng K trong trường hợp TNN cho kết quả chính xác khi so sánh $m_{r}$ khi có từ trường tăng $\approx 10\%$ với $m_{r}$ khi chưa có từ trường, điều này đã được đưa ra trong bài của Goryca \textit{et al.}\cite{goryca2019} trước đó.

\begin{figure}[htb]
	\begin{subfigure}{0.495\textwidth}
		\centering
		\includegraphics[width=\linewidth]{imgs/WS2/massK1.pdf}
	\end{subfigure}
	\begin{subfigure}{0.495\textwidth}
		\centering
		\includegraphics[width=\linewidth]{imgs/WS2/massK2.pdf}
	\end{subfigure}
	\caption{Khối lượng hiệu dụng.}
\end{figure}
\newpage
\subsubsection*{Monolayer WSe$_{2}$}
\begin{figure}[htb]
	\begin{subfigure}{0.495\textwidth}
		\centering
		\includegraphics[width=\linewidth]{imgs/WSe2/ButAndWave_c.pdf}
	\end{subfigure}
	\begin{subfigure}{0.495\textwidth}
		\centering
		\includegraphics[width=\linewidth]{imgs/WSe2/ButAndWave_v.pdf}
	\end{subfigure}
	\caption{Hàm sóng của 2 mức Landau kế tiếp nhau.}
\end{figure}
Cấu trúc dải năng lượng của WSe$_{2}$ khi chưa có từ trường cho thấy rằng điểm $\Gamma$ có mức rất thấp hơn điểm K, vậy nên khi áp dụng từ trường thì mức năng lượng của điểm $\Gamma$ này không xuất hiện ở đây. 

Khối lượng hiệu dụng cho WSe$_{2}$ khi chưa có từ trường được tính bằng công thức $\frac{1}{m_{ij}*} =\frac{1}{\hbar^{2}} \frac{\partial^{2} E(\mathbf{k})}{\partial k_{i} k_{j}}$ cho biết rằng $m_{e} \approx 0.3124 m_{0}, m_{h} \approx 0.4022 m_{0}$ và $m_{r} \approx 0.1758 m_{0}$ cho trường hợp TNN và $m_{e} \approx 0.3487 m_{0}, m_{h} \approx 0.4792 m_{0}$ và $m_{r} \approx 0.2018 m_{0}$ cho trường hợp NN, đã được đưa ra trong nghiên cứu của Kyl\"{a}np\"{a}\"{a} \textit{et al.} và Berkelbach \textit{et al.} \cite{kylanpaa2015,berkelbach2013}. \\
Khi ở từ trường lớn ví dụ như $B = 100 $T
\begin{itemize}
	\item[a)] Nearest neighbor
	\begin{itemize}
		\item Ở thung lũng K, $m_{h} \approx 0.5220 m_{0},\; m_{e} \approx 0.3702 m_{0}$. 
		Trong đó $m_{r} \approx 0.2166 m_{0}$ tăng $\approx 7.34\%$.
		
		\item Ở thung lũng K', $m_{h} \approx 0.4888 m_{0},\; m_{e} \approx 0.3573 m_{0}$. 
		Trong đó $m_{r} \approx 0.2064 m_{0}$ tăng $\approx 2.28\%$.
	\end{itemize}
	\item[b)] Third nearest neighbor
	\begin{itemize}
		\item Ở thung lũng K, $m_{h} \approx 0.4417 m_{0},\; m_{e} \approx 0.3494 m_{0}$. 
		Trong đó $m_{r} \approx 0.1951 m_{0}$ tăng $\approx 10.98\%$.
		
		\item Ở thung lũng K', $m_{h} \approx 0.4257 m_{0},\; m_{e} \approx 0.3199 m_{0}$. 
		Trong đó $m_{r} \approx 0.1826 m_{0}$ tăng $\approx 3.87\%$.
	\end{itemize}
\end{itemize}
Trong bài nghiên cứu của Stier \textit{et al.} \cite{stier2018} đã đưa ra rằng, $m_{r}$ sẽ lớn hơn sấp sĩ 15\% so với dự đoán của những lý thuyết gần đây \cite{berkelbach2013,kylanpaa2015}, với $m_{r} \approx 0.20 \pm 0.01 m_{0}$.  Như vậy ở thung lũng K trong trường hợp TNN cho kết quả chính xác khi so sánh $m_{r}$ khi có từ trường tăng $\approx 11\%$ với $m_{r}$ khi chưa có từ trường, điều này đã được đưa ra trong bài của Stier \textit{et al.}\cite{stier2018} trước đó.
\begin{figure}[htb]
	\begin{subfigure}{0.495\textwidth}
		\centering
		\includegraphics[width=\linewidth]{imgs/WSe2/massK1.pdf}
	\end{subfigure}
	\begin{subfigure}{0.495\textwidth}
		\centering
		\includegraphics[width=\linewidth]{imgs/WSe2/massK2.pdf}
	\end{subfigure}
	\caption{Khối lượng hiệu dụng.}
\end{figure}


\newpage
\subsubsection*{Monolayer WTe$_{2}$}
\begin{figure}[htb]
	\begin{subfigure}{0.495\textwidth}
		\centering
		\includegraphics[width=\linewidth]{imgs/WTe2/ButAndWave_c.pdf}
	\end{subfigure}
	\begin{subfigure}{0.495\textwidth}
		\centering
		\includegraphics[width=\linewidth]{imgs/WTe2/ButAndWave_v.pdf}
	\end{subfigure}
	\caption{Hàm sóng của 2 mức Landau kế tiếp nhau.}
\end{figure}

Cấu trúc dải năng lượng của WTe$_{2}$ khi chưa có từ trường cho thấy rằng điểm $\Gamma$ có mức rất thấp hơn điểm K, vậy nên khi áp dụng từ trường thì mức năng lượng của điểm $\Gamma$ này không xuất hiện ở đây.  

Khối lượng hiệu dụng cho WTe$_{2}$ khi chưa có từ trường được tính bằng công thức $\frac{1}{m_{ij}*} =\frac{1}{\hbar^{2}} \frac{\partial^{2} E(\mathbf{k})}{\partial k_{i} k_{j}}$ cho biết rằng $m_{e} \approx 0.2478 m_{0}, m_{h} \approx 0.3332 m_{0}$ và $m_{r} \approx 0.1421 m_{0}$ cho trường hợp TNN và $m_{e} \approx 0.3169 m_{0}, m_{h} \approx 0.4559 m_{0}$ và $m_{r} \approx 0.187 m_{0}$ cho trường hợp NN. Cho tới nay thì chưa tìm thấy được bài nào có nói về khối lượng hiệu dụng WTe$_{2}$. \\
Khi ở từ trường lớn ví dụ như $B = 80 $T
\begin{itemize}
	\item[a)] Nearest neighbor
	\begin{itemize}
		\item Ở thung lũng K, $m_{h} \approx 0.5093 m_{0},\; m_{e} \approx 0.3413 m_{0}$. 
		Trong đó $m_{r} \approx 0.2044 m_{0}$ tăng $\approx 9.3\%$.
		
		\item Ở thung lũng K', $m_{h} \approx 0.4681 m_{0},\; m_{e} \approx 0.3264 m_{0}$. 
		Trong đó $m_{r} \approx 0.1923 m_{0}$ tăng $\approx 2.83\%$.
	\end{itemize}
	\item[b)] Third nearest neighbor
	\begin{itemize}
		\item Ở thung lũng K, $m_{h} \approx 0.387 m_{0},\; m_{e} \approx 0.2824 m_{0}$. 
		Trong đó $m_{r} \approx 0.1633 m_{0}$ tăng $\approx 14.92\%$.
		
		\item Ở thung lũng K', $m_{h} \approx 0.3562 m_{0},\; m_{e} \approx 0.2577 m_{0}$. 
		Trong đó $m_{r} \approx 0.1495 m_{0}$ tăng $\approx 5.21\%$.
	\end{itemize}
\end{itemize}
Như vậy ở thung lũng K trong trường hợp TNN cho kết quả khối lượng rút gọn $m_r$ của WTe$_{2}$ là cho kết quả tăng khoảng 15\%, tương tự với các vật liệu trên.
\begin{figure}[htb]
	\begin{subfigure}{0.495\textwidth}
		\centering
		\includegraphics[width=\linewidth]{imgs/WTe2/massK1.pdf}
	\end{subfigure}
	\begin{subfigure}{0.495\textwidth}
		\centering
		\includegraphics[width=\linewidth]{imgs/WTe2/massK2.pdf}
	\end{subfigure}
	\caption{Khối lượng hiệu dụng.}
\end{figure}

\newpage

\bibliographystyle{unsrt}
\bibliography{refs}

\end{document}
