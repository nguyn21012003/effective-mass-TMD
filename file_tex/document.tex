%\documentclass{article}
\documentclass{article}
\usepackage[utf8]{vietnam}
\usepackage[utf8]{inputenc}
\usepackage{anyfontsize,fontsize}
\changefontsize[13pt]{13pt}	
\usepackage{commath}
\usepackage{parskip,setspace}
\usepackage[table]{xcolor}
\usepackage{amssymb}
\usepackage[d]{esvect}
\usepackage[nottoc,notlot,notlof]{tocbibind}
\usepackage{slashed,cancel}
\usepackage{indentfirst,titlesec}
\usepackage{pdfpages}
\usepackage{graphicx,subcaption,floatrow,adjustbox,rotating}
\usepackage{nccmath}
\usepackage{mathtools}
\usepackage{amsfonts,esint}
\usepackage[printonlyused,withpage]{acronym}
\usepackage{wrapfig}
\usepackage[toc,page]{appendix}
\usepackage{amsmath,systeme}
\usepackage[thinc]{esdiff}
\usepackage{hyperref,tikz}
\usepackage{bm,physics,nicematrix}
\usepackage[numbers,comma,sort&compress]{natbib}
%footnote
\usepackage{fancyhdr}
\usepackage{array,tocbibind}
\usepackage{enumitem}
\pagestyle{empty}


\usepackage{geometry}
\geometry{
	a4paper,
	total={170mm,257mm},
	left=20mm,
	top=20mm,
}



\newcommand{\image}[1]{
	\begin{center}
		\includegraphics[width=0.5\textwidth]{pic/#1}
	\end{center}
}




\renewcommand{\l}{\ell}
\newcommand{\dps}{\displaystyle}

\newcommand{\f}[2]{\dfrac{#1}{#2}}
\newcommand{\at}[2]{\bigg\rvert_{#1}^{#2} }


\renewcommand{\baselinestretch}{1.5}



\hypersetup{
	colorlinks=true,
	linkcolor=black,
	filecolor=magenta,      
	urlcolor=cyan,
	pdftitle={UT},
	citecolor=green,
	pdfpagemode=FullScreen,
}

\urlstyle{same}

\titleformat{\chapter}[display]
{\centering\large\bfseries} 
{\textbf{\MakeUppercase{\chaptername}} \ \thechapter\vspace{15pt}}{20pt}
{\large} 

%\newcommand{\thesistitlee}{Three-band tight binding model for TMD monolayers in the presence of a magnetic field}
\newcommand{\thesistitlee}{title}
\newcommand{\address}{NATIONAL UNIVERSITY OF HO CHI MINH CITY UNIVERSITY OF SCIENCE}
\newcommand{\department}{FACULTY OF PHYSICS - ENGINEERING PHYSICS}
\newcommand{\thesisauthor}{Tran Khoi Nguyen}
\newcommand{\thesisadvisor}{Your Advisor's Name}
\newcommand{\graddate}{Ho Chi Minh City, 2025}
\newcommand{\thesisdedication}{To all the Ph.D. pursuing brave souls}

\newlist{abbrv}{itemize}{1}
\setlist[abbrv,1]{label=,labelwidth=1in,align=parleft,itemsep=0.1\baselineskip,leftmargin=!}
\begin{document}
%\setlength{\parindent}{20pt}
%\begin{titlepage}
%	\begin{center}
%		{\bfseries
%			
%			{\large {\bf \address}}\\
%			{{\textbf{\department}}}\\
%			{---------------------o0o--------------------}
%			\vspace{1.5cm}
%			
%			{\large {\bf UNDERGRADUATE THESIS}}\\
%			\vspace{3.0cm}
%			
%		}
%		
%	\end{center}
%	\textit{\textbf{\underline{Thesis title:}}}\\
%	\begin{center}
%		{\bfseries
%			
%			{\largerrr\thesistitlee}
%			\vspace{1in}
%			
%		}
%	\end{center}
%	\noindent
%	\makebox[\textwidth]{\hfill\makebox[3in]{\hrulefill}}\\
%	\begin{center}
%		\makebox[\textwidth]{\hfill\makebox[3in]{\hfill \textbf{Student: Tran Khoi Nguyen}\hfill}}
%		\makebox[\textwidth]{\hfill\makebox[3in]{\hfill \textbf{Supervisor: Dr. Huynh Thanh Duc}\hfill}}
%	\end{center}
%	\begin{tikzpicture}[remember picture, overlay]
%		\draw[line width=2pt]
%		([xshift=1.5cm, yshift=1.5cm] current page.south west)
%		rectangle
%		([xshift=-1.5cm, yshift=-1.5cm] current page.north east);
%	\end{tikzpicture}
%	\begin{center}
%		\vspace{2.5in}
%		{\graddate}
%	\end{center}
%\end{titlepage}
%\begin{titlepage}
%	\begin{center}
%		{\bfseries
%			
%			{\large {\bf \address}}\\
%			{{\textbf{\department}}}\\
%			\vspace{2.5cm}
%			
%			{\large {\bf UNDERGRADUATE THESIS}}\\
%			\vspace{3.0cm}
%			
%		}
%		
%	\end{center}
%	\textit{\textbf{\underline{Thesis title:}}}\\
%	\begin{center}
%		{\bfseries
%			
%			{\largerrr\thesistitlee}
%			\vspace{1in}
%			
%		}
%	\end{center}
%	\noindent
%	\makebox[\textwidth]{\hfill\makebox[3in]{\hrulefill}}\\
%	\begin{center}
%		\makebox[\textwidth]{\hfill\makebox[3in]{\hfill \textbf{Student: Tran Khoi Nguyen}\hfill}}
%		\makebox[\textwidth]{\hfill\makebox[3in]{\hfill \textbf{Supervisor: Dr. Huynh Thanh Duc}\hfill}}
%	\end{center}
%	\begin{center}
%		\vspace{2.5in}
%		{\graddate}
%	\end{center}
%\end{titlepage}
%
%\newpage
%\pagenumbering{roman}
%\pagestyle{fancy}
%\renewcommand{\headrulewidth}{0pt}
%\fancyhf{}
%\fancyfoot[C]{\hspace{0cm} \thepage}
%\setcounter{page}{1}
%\pagenumbering{arabic}
\section{Solution}
%\noindent In the presence of a magnetic field, the vector potential induces the Peierls phases to accumulate in the Bloch wave functions. A flux quantum $\Phi$ over a triangular TMDC's corresponds to a magnetic field of the order of 46,928 T. For smaller $\Phi_{0}$, the unit cell has contain $2q$ atoms and the Hamiltonian becomes an $6q \times 6q$ matrix. However the magnetic field $B\approx 500$~T in the calculation is extremely large compared to the maximum experimental value $B \approx 50$~T, so we need to have another way to diagonalize the Hamiltonian. For the field reachable in experimetnts, it requires huge computing time to solve the Hamiltonian.
In the case $B \neq 0$, wave function can be expressed as the coefficients of $C_{ji}^{\lambda}$ in the tight-binding wave function
\begin{gather}
	\psi_{\lambda,\mathbf{k}}(\mathbf{r}) = \sum_{j}^{3}\sum_{i}^{2q} C_{ji}^{\lambda}(\mathbf{k}) \sum_{{\alpha}}^{N_{\text{UC}}} e^{i\mathbf{k}\cdot(\mathbf{R}_{\alpha} + \mathbf{r}_{i})} \phi_{j}(\mathbf{r} - \mathbf{R}_{\alpha} - \mathbf{r}_{i}).
\end{gather}
where $i = 1 ... 2q$. We have shown that, under an uniform magnetic field, Bloch bands $\lambda$ construct Landau levels at small fields and become fractal-structured at strong fields, which is known as the Hofstadter butterfly.

%\begin{figure*}[htb]
%	\centering
%	\includegraphics[width=\linewidth]{imgs/plotWaveFunction.pdf}
%	\caption{band $d_{z^{2}}$}
%\end{figure*}
%\begin{figure*}[htb]
%	\centering
%	\includegraphics[width=0.9\linewidth]{imgs/plotWaveFunctiond2.pdf}
%	\caption{band $d_{xy}$}
%\end{figure*}
%\begin{figure*}[htb]
%	\centering
%	\includegraphics[width=0.9\linewidth]{imgs/plotWaveFunctiond3.pdf}
%	\caption{band $d_{x^{2} - y^{2}}$}
%\end{figure*}
%\newpage
We now consider a new basis consisting of three eigenfunctions of the angular momentum operators $L^2$ and $L_{z}$, for $l = 2, m = 0, \pm 2$,
\begin{gather}
	\ket{\tilde{\phi}_{1}} = \ket{d_{m = 0}}, \quad
	\ket{\tilde{\phi}_{2}} = \ket{d_{m = +2}}, \quad
	\ket{\tilde{\phi}_{3}} = \ket{d_{m = -2}}.
\end{gather}
The new basis can be obtained from the old one by the transformation
\begin{gather}
	W =
	\begin{pNiceMatrix}
		1 & 0                  & 0                   \\
		0 & \frac{i}{\sqrt{2}} & \frac{1}{\sqrt{2}}  \\
		0 & -\frac{i}{\sqrt{2}} & \frac{1}{\sqrt{2}}
	\end{pNiceMatrix}.
\end{gather}
The TB Hamiltonian in new basis reads
\begin{equation}
	\begin{aligned}
		\tilde{H}^{\text{NN}} (\mathbf{k}) & = W H^{NN}(\mathbf{k}) W^{\dagger} \\
		                                   & =
		\begin{pNiceMatrix}
			h_{0}                                        & \frac{1}{\sqrt{2}} (h_{1} - i h_{2})          & \frac{1}{\sqrt{2}} (h_{1} + i h_{2})           \\
			\frac{1}{\sqrt{2}} (h_{1}^{*} + i h_{2}^{*}) & \frac{1}{2} (h_{11} + h_{22} + 2 \Im{h_{12}}) & \frac{1}{2} (h_{11} - h_{22} + 2i \Re{h_{12}}) \\
			\frac{1}{\sqrt{2}} (h_{1}^{*} - i h_{2}^{*}) & \frac{1}{2} (h_{11} - h_{22} - 2 \Im{h_{12}}) & \frac{1}{2} (h_{11} + h_{22} - 2i \Re{h_{12}})
		\end{pNiceMatrix}.
	\end{aligned}
\end{equation}
\begin{figure*}[htb]
	\centering
	\includegraphics[width=0.75\linewidth]{imgs/BSDFT.pdf}
	\caption{Bandstructure lấy từ bài Liu.}
\end{figure*}
\begin{figure*}[htb]
	\centering
	\includegraphics[width=0.75\linewidth]{imgs/NN-Tran-d0-d1-d2-2q-2q1.pdf}
	\caption{Wavefunctions using NN model.}
\end{figure*}
%\begin{figure*}[htb]
%	\centering
%	\includegraphics[width=\linewidth]{imgs/TNN-Tran-d0-d1-d2-2q-2q1.pdf}
%	\caption{Wavefunctions using TNN model.}
%\end{figure*}
Ở {Hình~1}, tại điểm $K$, dải hoá trị có đóng góp chủ yếu từ hai orbital $d_{xy}$ và $d_{x^{2} - y^{2}}$, trong khi dải dẫn chủ yếu đến từ orbital $d_{z^{2}}$. Khi có mặt từ trường, vùng Brillouin zone bị thu nhỏ lại $2q$ lần so với vùng Brillouin gốc, khiến các điểm $K$ và $K'$ co lại và nằm gần điểm $\Gamma$. Do đó, trong bài này, chúng tôi chỉ xét phổ Hofstadter butterfly tại điểm $\Gamma$.

Ở {Hình~2}, khi có từ trường, hàm sóng được biểu diễn tại $\Gamma = (0,0)$. Cụ thể, các hình {(a)}, {(b)}, {(c)} tương ứng với mức $2q$ (dải hoá trị), trong khi các hình {(d)}, {(e)}, {(f)} tương ứng với mức $2q+1$ (dải dẫn).

\subsection{Cyclotron frequency}
Tại $p = 1$, $q = 4723$, tần số cyclotron được tính theo công thức
\begin{equation}
	\begin{aligned}
		\hbar \omega_{c}       & = E_{n+1} - E_{n}                \\
		\Rightarrow \omega_{c} & = \frac{E_{n+1} - E_{n}}{\hbar},
	\end{aligned}
\end{equation}
và khối lượng hiệu dụng cyclotrong được tính bằng công thức
\begin{equation}
	\begin{aligned}
		\omega_{c}        & = \frac{eB}{m^{*}}
		\Rightarrow m^{*} & =  \frac{eB}{\omega_{c}} = \frac{eB}{\frac{E_{n+1} - E_{n}}{\hbar}} = \frac{eB \hbar}{E_{n+1} - E_{n}}
	\end{aligned}
\end{equation}
trong đó $n$ là chỉ số mức Landau. Hàm sóng của 2 mức Landau kế tiếp nhau ở điểm $K$ được thể hiện qua Fig.3. Ở hình 3(a),(b),(c) là hàm sóng ở mức Landau $n=1$, với dải $2q+4$, hình 3(d),(e),(f) là hàm sóng ở mức Landau $n=2$ với dải $2q+8$
\begin{figure}[htb]
	\begin{subfigure}{0.495\textwidth}
		\centering
		\includegraphics[width=\linewidth]{imgs/ButAndWave_c.pdf}
	\end{subfigure}
	\begin{subfigure}{0.495\textwidth}
		\centering
		\includegraphics[width=\linewidth]{imgs/ButAndWave_v.pdf}
	\end{subfigure}
	\caption{Hàm sóng của 2 mức Landau kế tiếp nhau.}
\end{figure}

\begin{table}[h]
	\label{Table 2.1}
	\begin{equation*}
		\renewcommand{\arraystretch}{1.5}
		\begin{NiceArray}{c c | c c}
			\hline
			\hline
			\text{Band} \; \lambda & \text{Label} & \text{Band} \; \lambda & \text{Label}  \\
			\hline
			2q+1                   & \ev{0,0}_{K'} & 2q+7                   & \ev{0,2}_{K}  \\
			2q+2                   & \ev{0,0}_{K'} & 2q+8                   & \ev{0,2}_{K}  \\
			2q+3                   & \ev{0,1}_{K}  & 2q+9                   & \ev{0,1}_{K'} \\
			2q+4                   & \ev{0,1}_{K}  & 2q+10                  & \ev{0,1}_{K'} \\
			2q+5                   & \ev{0,0}_{K'} & 2q+11                  & \ev{0,2}_{K}  \\
			2q+6                   & \ev{0,0}_{K'} & 2q+12                  & \ev{0,3}_{K}  \\
			\hline
			\hline
		\end{NiceArray}
	\end{equation*}
	\caption{Dán nhãn cho từng band $\lambda$.}
\end{table}
Từ Eq.(6) ta tính ra được khối lượng hiệu dụng $m^{*}/m_{0}$ và tần số cyclotron, được biểu diễn trong hình.4 và hình.5
%\begin{figure*}[htb]
%	\centering
%	\includegraphics[width=0.5\linewidth]{imgs/meff1.pdf}
%	\caption{Khối lượng hiệu dụng $m*/m_{0}$ tại $q = 4723$.}
%\end{figure*}
%\begin{figure*}[htb]
%	\centering
%	\includegraphics[width=0.5\linewidth]{imgs/meffTNN.pdf}
%	\caption{Khối lượng hiệu dụng $m*/m_{0}$ tại $q = 4723$.}
%\end{figure*}
%trong đó $m_{0}$ là khối lượng electron.

Ở hình.4 khối lượng hiệu dụng ở $B=40$~T đột ngột tăng lên bất thường là do năng lượng ở 2 mức Landau kết tiếp nhau $\Delta E \approxeq 0$ dẫn đến tăng lên đột ngột. Điều này có thể thấy ở hình.5 tần số Cyloctron giảm từ $\approx 3\times10^{13}$ xuống còn $5\times10^{12}$ Hz

\newpage


\begin{table}[h]
	\begin{equation*}
		\renewcommand{\arraystretch}{1.5}
		\begin{NiceArray}{c c}
			\hline
			\hline
			\text{Band} \; \lambda & \text{Label} \\
			\hline
			2q - 1 & \ev{2,0}_{K'} \\
			2q - 2 & \ev{2,0}_{K'} \\
			2q - 3 & \ev{0,0}_{\Gamma} \\
			2q - 4 & \ev{0,0}_{\Gamma} \\
			2q - 5 & \ev{0,1}_{\Gamma} \\
			2q - 6 & \ev{0,1}_{\Gamma} \\
			2q - 7 & \ev{2,1}_{K'} \\
			2q - 8 & \ev{2,1}_{K'} \\
			2q - 9 & \ev{2,2}_{\Gamma} \\
			2q - 10 & \ev{0,2}_{\Gamma} \\
			2q - 11 & \ev{0,3}_{\Gamma} \\
			2q - 12 & \ev{0,3}_{\Gamma} \\
			2q - 13 & \ev{0,4}_{\Gamma} \\
			2q - 14 & \ev{0,4}_{\Gamma} \\
			\cellcolor{yellow} 2q - 15 & \cellcolor{yellow} \ev{2,0}_{K} \\
			\cellcolor{yellow} 2q - 16 & \cellcolor{yellow} \ev{2,0}_{K} \\
			2q - 17 & \ev{2,2}_{K'} \\
			2q - 18 & \ev{2,2}_{K'} \\
			2q - 19 & \ev{0,5}_{\Gamma} \\
			2q - 20 & \ev{0,5}_{\Gamma} \\
			2q - 21 & \ev{0,6}_{\Gamma} \\
			2q - 22 & \ev{0,6}_{\Gamma} \\
			2q - 23 & \ev{0,7}_{\Gamma} \\
			2q - 24 & \ev{0,7}_{\Gamma} \\
			2q - 25 & \ev{2,3}_{K'} \\
			2q - 26 & \ev{2,3}_{K'} \\
			\cellcolor{yellow} 2q - 27 & \cellcolor{yellow} \ev{2,1}_{K} \\
			\cellcolor{yellow} 2q - 28 & \cellcolor{yellow} \ev{2,1}_{K} \\
			2q - 29 & \ev{0,8}_{\Gamma} \\
			2q - 30 & \ev{0,8}_{\Gamma} \\
			\hline
			\hline
		\end{NiceArray}
	\end{equation*}
	\caption{Dán nhãn cho từng band $\lambda$.}
\end{table}

\begin{table}[h]
	\begin{equation*}
		\renewcommand{\arraystretch}{1.5}
		\begin{NiceArray}{c c}
			\hline
			\hline
			\text{Band} \; \lambda & \text{Label} \\
			\hline
			2q + 1 & \ev{0,0}_{K'}\\
			2q + 1 & \ev{0,0}_{K'}\\
			\hline
			\hline
		\end{NiceArray}
	\end{equation*}
	\caption{Dán nhãn cho từng band $\lambda$.}
\end{table}





\end{document}