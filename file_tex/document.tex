%\documentclass{article}
\documentclass{article}
\usepackage[utf8]{vietnam}
\usepackage[utf8]{inputenc}
\usepackage{anyfontsize,fontsize}
\changefontsize[13pt]{13pt}	
\usepackage{commath}
\usepackage{parskip,setspace}
\usepackage[table]{xcolor}
\usepackage{amssymb}
\usepackage[d]{esvect}
\usepackage[nottoc,notlot,notlof]{tocbibind}
\usepackage{slashed,cancel}
\usepackage{indentfirst,titlesec}
\usepackage{pdfpages}
\usepackage{graphicx,subcaption,floatrow,adjustbox,rotating}
\usepackage{nccmath}
\usepackage{mathtools}
\usepackage{amsfonts,esint}
\usepackage[printonlyused,withpage]{acronym}
\usepackage{wrapfig}
\usepackage[toc,page]{appendix}
\usepackage{amsmath,systeme}
\usepackage[thinc]{esdiff}
\usepackage{hyperref,tikz}
\usepackage{bm,physics,nicematrix}
% \usepackage[numbers,comma,sort&compress]{natbib}
%footnote
\usepackage{fancyhdr}
\usepackage{array,tocbibind}
\usepackage{enumitem}
\pagestyle{empty}


\usepackage{geometry}
\geometry{
	a4paper,
	total={170mm,257mm},
	left=20mm,
	top=20mm,
}



\newcommand{\image}[1]{
	\begin{center}
		\includegraphics[width=0.5\textwidth]{pic/#1}
	\end{center}
}




\renewcommand{\l}{\ell}
\newcommand{\dps}{\displaystyle}

\newcommand{\f}[2]{\dfrac{#1}{#2}}
\newcommand{\at}[2]{\bigg\rvert_{#1}^{#2} }


\renewcommand{\baselinestretch}{1.5}



\hypersetup{
	colorlinks=true,
	linkcolor=black,
	filecolor=magenta,      
	urlcolor=cyan,
	pdftitle={UT},
	citecolor=black,
	pdfpagemode=FullScreen,
}

\urlstyle{same}

\titleformat{\chapter}[display]
{\centering\large\bfseries} 
{\textbf{\MakeUppercase{\chaptername}} \ \thechapter\vspace{15pt}}{20pt}
{\large} 

%\newcommand{\thesistitlee}{Three-band tight binding model for TMD monolayers in the presence of a magnetic field}
\newcommand{\thesistitlee}{title}
\newcommand{\address}{NATIONAL UNIVERSITY OF HO CHI MINH CITY UNIVERSITY OF SCIENCE}
\newcommand{\department}{FACULTY OF PHYSICS - ENGINEERING PHYSICS}
\newcommand{\thesisauthor}{Tran Khoi Nguyen}
\newcommand{\thesisadvisor}{Your Advisor's Name}
\newcommand{\graddate}{Ho Chi Minh City, 2025}
\newcommand{\thesisdedication}{To all the Ph.D. pursuing brave souls}

\newlist{abbrv}{itemize}{1}
\setlist[abbrv,1]{label=,labelwidth=1in,align=parleft,itemsep=0.1\baselineskip,leftmargin=!}
\begin{document}
\setlength{\parindent}{20pt}
%\begin{titlepage}
%	\begin{center}
%		{\bfseries
%			
%			{\large {\bf \address}}\\
%			{{\textbf{\department}}}\\
%			{---------------------o0o--------------------}
%			\vspace{1.5cm}
%			
%			{\large {\bf UNDERGRADUATE THESIS}}\\
%			\vspace{3.0cm}
%			
%		}
%		
%	\end{center}
%	\textit{\textbf{\underline{Thesis title:}}}\\
%	\begin{center}
%		{\bfseries
%			
%			{\largerrr\thesistitlee}
%			\vspace{1in}
%			
%		}
%	\end{center}
%	\noindent
%	\makebox[\textwidth]{\hfill\makebox[3in]{\hrulefill}}\\
%	\begin{center}
%		\makebox[\textwidth]{\hfill\makebox[3in]{\hfill \textbf{Student: Tran Khoi Nguyen}\hfill}}
%		\makebox[\textwidth]{\hfill\makebox[3in]{\hfill \textbf{Supervisor: Dr. Huynh Thanh Duc}\hfill}}
%	\end{center}
%	\begin{tikzpicture}[remember picture, overlay]
%		\draw[line width=2pt]
%		([xshift=1.5cm, yshift=1.5cm] current page.south west)
%		rectangle
%		([xshift=-1.5cm, yshift=-1.5cm] current page.north east);
%	\end{tikzpicture}
%	\begin{center}
%		\vspace{2.5in}
%		{\graddate}
%	\end{center}
%\end{titlepage}
%\begin{titlepage}
%	\begin{center}
%		{\bfseries
%			
%			{\large {\bf \address}}\\
%			{{\textbf{\department}}}\\
%			\vspace{2.5cm}
%			
%			{\large {\bf UNDERGRADUATE THESIS}}\\
%			\vspace{3.0cm}
%			
%		}
%		
%	\end{center}
%	\textit{\textbf{\underline{Thesis title:}}}\\
%	\begin{center}
%		{\bfseries
%			
%			{\largerrr\thesistitlee}
%			\vspace{1in}
%			
%		}
%	\end{center}
%	\noindent
%	\makebox[\textwidth]{\hfill\makebox[3in]{\hrulefill}}\\
%	\begin{center}
%		\makebox[\textwidth]{\hfill\makebox[3in]{\hfill \textbf{Student: Tran Khoi Nguyen}\hfill}}
%		\makebox[\textwidth]{\hfill\makebox[3in]{\hfill \textbf{Supervisor: Dr. Huynh Thanh Duc}\hfill}}
%	\end{center}
%	\begin{center}
%		\vspace{2.5in}
%		{\graddate}
%	\end{center}
%\end{titlepage}
%
%\newpage
\pagenumbering{roman}
\pagestyle{fancy}
\renewcommand{\headrulewidth}{0pt}
\fancyhf{}
\fancyfoot[C]{\hspace{0cm} \thepage}
\setcounter{page}{1}
\pagenumbering{arabic}
\section{Theory}
\subsection{Three-band tight-binding model}
\noindent In the model introduced by Liu~\textit{et al.}, only the orbitals of the M atom are included. We denote the wave functions of the three orbitals of the M atom as
\begin{gather}
	\ket{\phi_{1}} = \ket{d_{z^{2}}} , \quad \ket{\phi_{2}} = \ket{d_{xy}} , \quad \ket{\phi_{3}} = \ket{d_{x^{2} - y^{2}}}.
\end{gather}
The Bloch wavefunction in this model has the form
\begin{gather}
	\psi_{\mathbf{k}}^{\lambda} (\mathbf{r}) = \sum_{j=1}^{3} C_{j}^{\lambda}(\mathbf{k}) \sum_{\mathbf{R}} e^{i \mathbf{k \cdot R}} \phi_{j}(\mathbf{r} - \mathbf{R}).
\end{gather}
The coefficents $C_{j}^{\lambda}(\mathbf{k})$ are the solutions of the eigenvalue equation
\begin{gather}
	\sum_{jj'}^{3} \left[H_{jj'}^{\text{TB}}(\mathbf{k}) - \varepsilon_{\lambda}(\mathbf{k}) S_{jj'}(\mathbf{k})\right] C_{j}^{\lambda}(\mathbf{k}) = 0,
\end{gather}
where
\begin{equation}
	\begin{aligned}
		H_{jj'}^{\text{TB}}(\mathbf{k}) = \sum_{\mathbf{R}} e^{i \mathbf{k \cdot R}} \bra{\phi_{j}(\mathbf{r})} H_{\text{1e}} \ket{\phi_{j'}(\mathbf{r - R})},
	\end{aligned}
\end{equation}
and
\begin{equation}
	\begin{aligned}
		S_{jj'}(\mathbf{k}) = \sum_{\mathbf{R}} \bra{\phi_{j}(\mathbf{r})} \ket{\phi_{j'}(\mathbf{r - R})} \approx \delta_{jj'}.
	\end{aligned}
\end{equation}

In the case $B \neq 0$, the new lattice vector now is $\mathbf{R} = k \mathbf{a}_{1} + l 2q \mathbf{a}_{2}$, where $k,l \in \mathbb{Z}$. The wavefunction has an additional phase factor
\begin{gather}
	\psi_{\mathbf{k}}^{\lambda} (\mathbf{r}) = \sum_{j=1}^{3} C_{j}^{\lambda} \sum_{\mathbf{R}} e^{i \mathbf{k} \cdot \mathbf{R}} e^{i \theta_{\mathbf{R}} (\mathbf{r})} \phi_{j} (\mathbf{r} - \mathbf{R}),
\end{gather}
and choose $\theta = -\frac{e}{\hbar}\int_{\mathbf{r}}^{\mathbf{R}} \mathbf{A}(\mathbf{r}') \cdot d \mathbf{r}'$ as Peierls phase factor, the Hamiltonian now is
\begin{gather}
	H_{j  j' }^{} = \sum_{\mathbf{R}} e^{i \mathbf{k} \cdot \mathbf{R}} e^{\frac{ie}{\hbar} \int_{0}^{\mathbf{R}} \mathbf{A}(\mathbf{r}) \cdot d \mathbf{r}} E_{jj'} (\mathbf{R}),
\end{gather}
where 
\begin{gather}
	E_{jj'} = \bra{\phi_{j}(\mathbf{r})} H_{1e} \ket{\phi_{j'} (\mathbf{r} - \mathbf{R})}.
\end{gather}
Using a uniform magnetic field $\mathbf{B} = (0,0,B)$ and Landau gauge $\mathbf{A} = (By,0,0)$. The Peierls hopping phase is given
\begin{equation}
	\begin{aligned}
		\frac{ie}{\hbar} \int_{\mathbf{0}}^{\mathbf{R}} \mathbf{A}(\mathbf{r}) \cdot d \mathbf{r}& = \frac{ie}{\hbar} \int_{\mathbf{0}}^{\mathbf{R}} B y dx \\
		&= \frac{ieB}{\hbar} \int_{0}^{1} y(\tau) x'(\tau) d\tau,
	\end{aligned}
\end{equation}
suppose that the atom M is located at lattice vector $\mathbf{R}_{m,n}$, the Peierls phase can be written as
\begin{gather}
	\theta_{m,n}^{m',n'} =
	\begin{cases}
		0                                                                             & \quad m' = m \pm 2, n' = n  ,      \\
		0                                                                             & \quad m' = m \pm 4, n' = n  ,      \\
		\pm \frac{e}{\hbar} \frac{B a^{2} \sqrt{3}}{2} m                      & \quad m' = m , n' = n \pm 2,  \\
		\pm \frac{e}{\hbar} \frac{B a^{2} \sqrt{3}}{4} \left(m \mp \frac{1}{2}\right) & \quad m' = m \mp 1, n' = n \pm 1 , \\
		\pm \frac{e}{\hbar} \frac{B a^{2} \sqrt{3}}{2} (m \mp 1)                      & \quad m' = m \mp 2, n' = n \pm 2,  \\
		\pm \frac{e}{\hbar} \frac{B a^{2} \sqrt{3}}{4} \left(m \mp \frac{3}{2}\right) & \quad m' = m \mp 3, n' = n \pm 1.  \\
	\end{cases}
\end{gather}
We obtain the Hamiltonian in magnetic field
\begin{equation}
	\begin{aligned}
		H_{jj'}^{\text{TB}}(\mathbf{k})
		& = E_{jj'}(\mathbf{0}) + e^{i \mathbf{k} \cdot \mathbf{R}_{1}} E_{jj'}(\mathbf{R}_{1}) 
		+ e^{-i\pi(m + 1/2)\tfrac{\Phi}{\Phi_{0}}} e^{i \mathbf{k} \cdot \mathbf{R}_{2}} E_{jj'}(\mathbf{R}_{2})  \\
		& + e^{-i\pi(m - 1/2)\tfrac{\Phi}{\Phi_{0}}} e^{i \mathbf{k} \cdot \mathbf{R}_{3}} E_{jj'}(\mathbf{R}_{3}) 
		+ e^{i \mathbf{k} \cdot \mathbf{R}_{4}} E_{jj'}(\mathbf{R}_{4})                                  \\
		& + e^{i\pi(m - 1/2)\tfrac{\Phi}{\Phi_{0}}} e^{i \mathbf{k} \cdot \mathbf{R}_{5}} E_{jj'}(\mathbf{R}_{5}) 
		+ e^{i\pi(m + 1/2)\tfrac{\Phi}{\Phi_{0}}} e^{i \mathbf{k} \cdot \mathbf{R}_{6}} E_{jj'}(\mathbf{R}_{6}) \\
		& + e^{- i\pi(m + 3/2)\tfrac{\Phi}{\Phi_{0}} } e^{i \mathbf{k} \cdot \mathbf{R}_{7}} E_{jj'}(\mathbf{R}_{7}) 
		+ e^{- 2i\pi m\tfrac{\Phi}{\Phi_{0}} } e^{i \mathbf{k} \cdot \mathbf{R}_{8}} E_{jj'}(\mathbf{R}_{8}) \\
		& + e^{- i\pi(m - 3/2)\tfrac{\Phi}{\Phi_{0}} } e^{i \mathbf{k} \cdot \mathbf{R}_{9}} E_{jj'}(\mathbf{R}_{9}) 
		+ e^{ i\pi (m-3/2)\tfrac{\Phi}{\Phi_{0}} } e^{i \mathbf{k} \cdot \mathbf{R}_{10}} E_{jj'}(\mathbf{R}_{10}) \\
		& + e^{2 i\pi m \tfrac{\Phi}{\Phi_{0}} } e^{i \mathbf{k} \cdot \mathbf{R}_{11}} E_{jj'}(\mathbf{R}_{11}) 
		+ e^{ i\pi (m+3/2)\tfrac{\Phi}{\Phi_{0}} } e^{i \mathbf{k} \cdot \mathbf{R}_{12}} E_{jj'}(\mathbf{R}_{12}) \\
		& + e^{i \mathbf{k} \cdot \mathbf{R}_{13}} E_{jj'}(\mathbf{R}_{13}) 
		+ e^{-2i\pi(m + 1)\tfrac{\Phi}{\Phi_{0}}} e^{i \mathbf{k} \cdot \mathbf{R}_{14}} E_{jj'}(\mathbf{R}_{14})  \\
		& + e^{-2i\pi(m - 1)\tfrac{\Phi}{\Phi_{0}}} e^{i \mathbf{k} \cdot \mathbf{R}_{15}} E_{jj'}(\mathbf{R}_{15}) 
		+ e^{i \mathbf{k} \cdot \mathbf{R}_{16}} E_{jj'}(\mathbf{R}_{16})                                  \\
		& + e^{2i\pi(m - 1)\tfrac{\Phi}{\Phi_{0}}} e^{i \mathbf{k} \cdot \mathbf{R}_{17}} E_{jj'}(\mathbf{R}_{17}) 
		+ e^{2i\pi(m + 1)\tfrac{\Phi}{\Phi_{0}}} e^{i \mathbf{k} \cdot \mathbf{R}_{18}} E_{jj'}(\mathbf{R}_{18}),
	\end{aligned}
\end{equation}
where $\Phi_{0} = \frac{h}{e}$ and $\Phi = \frac{\sqrt{3}}{2} Ba^{2}$.
Since the Peierls phase depends on the the atomic position specified by the site indices $m,n$, the Hamiltonian is no longer invariant under translation of a primitive vector. For the case $\frac{\Phi}{\Phi_{0}} = \frac{p}{q}$, with $p,q \in \mathbb{Z}$, it is possible to restore the translational invariance if we expand the unit cell so that it includes $2q$ M atoms. We, then, define a new basis set of $6q$ atomic orbitals $\left\{ \phi_{j}(\mathbf{r} - \mathbf{R}_{m,n}) \right\}$. The wave function can be expressed as the coefficients of $C_{ji}^{\lambda}$ in the tight-binding wave function
\begin{gather}
	\psi_{\mathbf{k}}^{\lambda}(\mathbf{r}) = \sum_{j}^{3}\sum_{i}^{2q} C_{ji}^{\lambda}(\mathbf{k}) \sum_{{\mathbf{R}}} e^{\frac{ie}{\hbar}\int_0^{\mathbf{R} + \mathbf{R}_{i}}\mathbf{A}(\mathbf{r})\cdot d\mathbf{r}} e^{i\mathbf{k} \cdot (\mathbf{R} + \mathbf{R}_{i}) } \phi_{j}(\mathbf{r} - \mathbf{R} - \mathbf{R}_{i}).
\end{gather}
where $j = 1,2,3$ and $i$ labels the atom $\mathbf{R}^{(i)}$ in the magnetic unit cell, with $i = 1, \ldots, 2q$. In this basis, the TB Hamiltonian has an additional Peierls phase
\begin{gather}
	H_{j i  j' i'} = \sum_{\mathbf{R}} e^{i \mathbf{k} \cdot ( \mathbf{R} - \mathbf{R}_{i} + \mathbf{R}_{i'} )} e^{\frac{ie}{\hbar} \int_{\mathbf{R}_{i}}^{\mathbf{R} + \mathbf{R}_{i'}} \mathbf{A}(\mathbf{r}) \cdot d \mathbf{r}} \bra{\phi_{j}(\mathbf{r} - \mathbf{R}_{i})} H_{1e} \ket{\phi_{j'}(\mathbf{r} - \mathbf{R} - \mathbf{R}_{i'})},
\end{gather}
The sum over $\mathbf{R}$ include up to third-nearest-neighbor hoppings. It is remarkbly to note that the lattice vectors satisfying the condition $\abs{\mathbf{R}} \leq 2a$ are $\mathbf{R} = \mathbf{0}, \pm \mathbf{a}_{1}, \pm 2 \mathbf{a}_{1}$, we obtain the Hamiltonian
\begin{equation}
	\begin{aligned}
		&H_{jnj'n'}^{\text{eff}}(\mathbf{k})
		= E_{jj'}(\mathbf{0}) \delta_{n,n'} 
		+ e^{i\mathbf{k} \cdot \mathbf{R}_{1}}E_{jj'}(\mathbf{R}_{1}) \delta_{n,n'} 
		+ e^{i\mathbf{k} \cdot \mathbf{R}_{4}}E_{jj'}(\mathbf{R}_{4}) \delta_{n,n'}  \\
		& + e^{-i\pi(m + 1/2)\frac{\Phi}{\Phi_{0}}}e^{i\mathbf{k} \cdot \mathbf{R}_{2}} E_{jj'}(\mathbf{R}_{2}) \delta_{n-1,n'} 
		+ e^{-i\pi(m - 1/2)\frac{\Phi}{\Phi_{0}}} e^{i\mathbf{k} \cdot \mathbf{R}_{3}} E_{jj'}(\mathbf{R}_{3}) \delta_{n-1,n'} \\
		& + e^{i\pi(m - 1/2)\frac{\Phi}{\Phi_{0}}} e^{i\mathbf{k} \cdot \mathbf{R}_{5}} E_{jj'}(\mathbf{R}_{5}) \delta_{n+1,n'} 
		+ e^{i\pi(m + 1/2)\frac{\Phi}{\Phi_{0}}} e^{i\mathbf{k} \cdot \mathbf{R}_{6}} E_{jj'}(\mathbf{R}_{6}) \delta_{n+1,n'} \\
		& + e^{- i\pi(m + 3/2)\frac{\Phi}{\Phi_{0}} } e^{i \mathbf{k} \cdot \mathbf{R}_{7}} E_{jj'}(\mathbf{R}_{7}) \delta_{n-1,n'} 
		+ e^{- 2i\pi m\frac{\Phi}{\Phi_{0}} } e^{i \mathbf{k} \cdot \mathbf{R}_{8}} E_{jj'}(\mathbf{R}_{8}) \delta_{n-2,n'} \\
		& + e^{- i\pi(m - 3/2)\frac{\Phi}{\Phi_{0}} } e^{i \mathbf{k} \cdot \mathbf{R}_{9}} E_{jj'}(\mathbf{R}_{9}) \delta_{n-1,n'} 
		+ e^{ i\pi (m-3/2)\frac{\Phi}{\Phi_{0}} } e^{i \mathbf{k} \cdot \mathbf{R}_{10}} E_{jj'}(\mathbf{R}_{10}) \delta_{n+1,n'} \\
		& + e^{2 i\pi m \frac{\Phi}{\Phi_{0}} } e^{i \mathbf{k} \cdot \mathbf{R}_{11}} E_{jj'}(\mathbf{R}_{11}) \delta_{n+2,n'} 
		+ e^{ i\pi (m+3/2)\frac{\Phi}{\Phi_{0}} } e^{i \mathbf{k} \cdot \mathbf{R}_{12}} E_{jj'}(\mathbf{R}_{12}) \delta_{n+1,n'} \\
		& + e^{i \mathbf{k} \cdot \mathbf{R}_{13}} E_{jj'}(\mathbf{R}_{13}) \delta_{n,n'} 
		+ e^{-2i\pi(m + 1) \frac{\Phi}{\Phi_{0}} } e^{i \mathbf{k} \cdot \mathbf{R}_{14}} E_{jj'}(\mathbf{R}_{14}) \delta_{n-2,n'}  \\
		& + e^{-2i\pi(m - 1) \frac{\Phi}{\Phi_{0}}} e^{i \mathbf{k} \cdot \mathbf{R}_{15}} E_{jj'}(\mathbf{R}_{15}) \delta_{n-2,n'} 
		+ e^{i \mathbf{k} \cdot \mathbf{R}_{16}} E_{jj'}(\mathbf{R}_{16}) \delta_{n,n'}                                 \\
		& + e^{2i\pi(m - 1)\frac{\Phi}{\Phi_{0}}} e^{i \mathbf{k} \cdot \mathbf{R}_{17}} E_{jj'}(\mathbf{R}_{17}) \delta_{n+2,n'} 
		+ e^{2i\pi(m + 1)\frac{\Phi}{\Phi_{0}}} e^{i \mathbf{k} \cdot \mathbf{R}_{18}} E_{jj'}(\mathbf{R}_{18}) \delta_{n+2,n'}.
	\end{aligned}
\end{equation}
where $\Phi_{0} = \frac{h}{e}$, $\Phi = \frac{\sqrt{3}}{2} Ba^{2}$ and $E(\mathbf{R})$ are obtained from Liu \textit{et al.}

\subsection{The cyclotron theory}
The cyclotron frequency can be obtained from the energy difference between two Landau levels
\begin{equation}
	\begin{aligned}
		\hbar \omega_{c}       & = E_{n+1} - E_{n}  ,
	\end{aligned}
\end{equation}
which gives
\begin{equation}
	\begin{aligned}
		\omega_{c}  = \frac{E_{n+1} - E_{n}}{\hbar}.
	\end{aligned}
\end{equation}
On the other hand, the cyclotron frequency is also defined as
\begin{gather}
	\omega_{c}         = \frac{eB}{m^{*}}.
\end{gather}
Combining the two expressions, the effective mas can be written as 
\begin{gather}
	m^{*}  =  \frac{eB}{\omega_{c}} = \frac{eB}{\frac{E_{n+1} - E_{n}}{\hbar}} = \frac{eB \hbar}{E_{n+1} - E_{n}},
\end{gather}
and
\begin{gather}
	\omega_{c} = \frac{E_{n+1} - E_{n}}{\hbar}.
\end{gather}
The radius of cyclotron orbit can be written as
\begin{gather}
	r_{c} = \frac{v_{\perp}}{\omega_{c}} = \frac{v_{\perp} \hbar}{E_{n+1} - E_{n}} = \frac{v_{\perp} m^{*}}{eB},
\end{gather}
where $v_{\perp}$ is choosen to be equal to $3 \times 10^{7}$ m/s.




\section{Methods}

When a magnetic field is applied to the crystal lattice, the magnetic unit cell is enlarged $q$ times for square lattice ($2q$ times for hexagonal lattice). As a consequence, the magnetic Brillouin zone smaller $2q$ times than the orginal Brillouin zone.

In addition, the three bases $d_{z^{2}}, d_{xy}, d_{x^{2}-y^{2}}$, which were introduced by Liu \textit{et al.}, cannot clearly distinguish the $K$ and $K'$ points in the valence and conduction bands for two reasons. First, the squared amplitudes $|\psi|^{2}$ are identical. Second, in the magnetic Brillouin zone, the $K$ and $K'$ valleys cannot be intuitively distinguished by the dispersion relation $E(\mathbf{k})$; instead, one needs to examine the properties of the wave functions. Specifically, the electron wave function at the $K$ valley in conduction band is mainly contributed by $d_{z^{2}}$, while at the valence band it is $d_{xy} + d_{x^{2}-y^{2}}$. Furthemore, we can distingushed $K$ and $K'$ valleys at the valence by using bases
\begin{gather}
	\ket{\psi_{v}^{K}} = \frac{1}{\sqrt{2}} \left( \ket{d_{x^{2} - y^{2}}} + i \ket{d_{xy}} \right),\\
	\ket{\psi_{v}^{K'}} = \frac{1}{\sqrt{2}} \left( \ket{d_{x^{2} - y^{2}}} - i \ket{d_{xy}} \right).
\end{gather}
Therefore, it is necessary to adopt another basis set. We now consider a new basis consisting of the three eigenfunctions of the angular momentum operators $L^{2}$ and $L_{z}$, corresponding to $l = 2$ and $m = 0, \pm 2$.

\begin{gather}
	\ket{\tilde{\phi}_{1}} = \ket{d_{m = 0}}, \quad
	\ket{\tilde{\phi}_{2}} = \ket{d_{m = +2}}, \quad
	\ket{\tilde{\phi}_{3}} = \ket{d_{m = -2}}.
\end{gather}
The new basis can be obtained from the old one by the transformation
\begin{gather}
	\ket{\tilde{\phi}_{j}} = \sum_{j'} W_{j'j} \ket{\phi_{j'}},
\end{gather}
where
\begin{gather}
	W =
	\begin{pNiceMatrix}
		1 & 0                   & 0                  \\
		0 & \frac{i}{\sqrt{2}}  & -\frac{i}{\sqrt{2}} \\
		0 & \frac{1}{\sqrt{2}} & \frac{1}{\sqrt{2}}
	\end{pNiceMatrix}.
\end{gather}
In particular, 
\begin{gather}
	\ket{\tilde{\phi}_{1}} = \ket{\phi_{1}},\\
	\ket{\tilde{\phi}_{2}} = \frac{i}{\sqrt{2}} \ket{\phi_{2}} + \frac{1}{\sqrt{2}} \ket{\phi_{3}},\\
	\ket{\tilde{\phi}_{3}} = -\frac{i}{\sqrt{2}} \ket{\phi_{2}} + \frac{1}{\sqrt{2}} \ket{\phi_{3}}.
\end{gather}
The TB Hamiltonian in new basis reads
\begin{equation}
	\begin{aligned}
		\tilde{H}^{\text{TB}} (\mathbf{k}) & = W^{\dagger} H^{\text{TB}}(\mathbf{k}) W,
	\end{aligned}
\end{equation}
where $H^{\text{TB}} = H^{\text{NN}}$ or $H^{\text{TNN}}$.

To distinguish the states that originate from the original Brillouin zone, we follow the convention of Ho \textit{et al.}~\cite{ho2014}. Each Landau level is then labeled as $\ket{j, n}_{\tau}$, where $j$, $n$, and $\tau$ denote the orbital, Landau, and valley indices, respectively.

When diagonalizing the Hamiltonian in Eq.~(14), we obtain $2q$ eigenvalues for each orbital $\phi_{j}(\mathbf{r})$, with $j = 1,2,3$. In total, this gives $6q$ eigenvalues. The eigenvalues corresponding to the valence band range from $0$ to $2q$, while those of the conduction band range from $2q+1$ to $4q$. 

For instance, in Fig.~1(a), the first Landau level is labeled as $\ket{0,0}_{K'}$. This level is degenerate, i.e., $E_{2q+1} = E_{2q+2}$, corresponding to the same energy value. Similarly, the second Landau level is labeled as $\ket{0,1}_{K'}$, which corresponds to the two degenerate eigenvalues $E_{2q+3} = E_{2q+4}$, and so on for the subsequent Landau levels. In Fig.~1(e), for the valence band, we clearly observe the evidence of the Brillouin zone shrinking. At a given value of $B$, the energy value is obtained simultaneously at the $K$, $K'$ and $\Gamma$ points. The first Landau level in the valence band corresponds to the eigenvalue $E_{2q}$ (labeled as $\ket{0,0}_{\Gamma}$), which is degenerate with $E_{2q-1}$. The second Landau level then corresponds to $E_{2q-2}$ and $E_{2q-3}$, and so on for the lower levels.

In terms of Eq.~(22), we have
\begin{equation}
	\begin{aligned}
		m_{e}^{*} & = \frac{e B \hbar}{E_{2q+3} - E_{2q+1}}.
	\end{aligned}
\end{equation}
However, for $m_{h}^{*}$, the situation is quite different and more complicated than for $m_{e}^{*}$. For some cases, like MoS$_{2}$, this difficulty arises because the energy of the $\Gamma$ point lies close to the $K$ and $K'$ valleys. 
As a consequence, in Fig.~1(d), the eigenvalues are no longer linear or follow the sequential index $2q-n$ as in the case of $m_{e}^{*}$, but instead exhibit level crossings due to numerical issues. 
To address this, first, we need to determine the energy value $E_{n}$ at a given $B$ that corresponds to the envelope function. 
This can be done by plotting all the wave functions from $0$ to $2q$, since each wave function provides information about the Landau level labeling. 
From the wave functions, we can then identify which $E_{n}$ corresponds to $\ket{j,n}_{\tau}$.



\newpage
\section{Numerical results}
\subsection{Effective mass}
\subsubsection*{Monolayer MoS$_{2}$}

\begin{figure}[htb]
	\begin{subfigure}{0.495\textwidth}
		\centering
		\includegraphics[width=\linewidth]{imgs/MoS2/ButAndWave_c.pdf}
	\end{subfigure}
	\begin{subfigure}{0.495\textwidth}
		\centering
		\includegraphics[width=\linewidth]{imgs/MoS2/ButAndWave_v.pdf}
	\end{subfigure}
	\caption{Landau levels (a) and the corresponding envelope-function components (b),(c),(d) for conduction electrons at valleys $K$ and $K'$. Figs (e)–(h) show the same as (a)–(d) but for valence electrons. (Recalculated from Ho \textit{et al.} \cite{ho2014})}
\end{figure}

The band structure of MoS$_{2}$ without a magnetic field shows that, in the valence band, the $\Gamma$ point has an energy level of $E \approx -0.058$ eV. Therefore, when a magnetic field is applied, this $\Gamma$-point energy level still appears.  

The effective masses of MoS$_{2}$ in the absence of a magnetic field, calculated from 
\begin{gather}
	\frac{1}{m_{ij}^{*}} = \frac{1}{\hbar^{2}} \frac{\partial^{2} E(\mathbf{k})}{\partial k_{i}\partial k_{j}},
\end{gather}
are $m_{e} \approx 0.4178 m_{0}$, $m_{h} \approx 0.5325 m_{0}$, and $m_{r} \approx 0.2341$ for the TNN case, and $m_{e} \approx 0.4508 m_{0}$, $m_{h} \approx 0.6487 m_{0}$, and $m_{r} \approx 0.2659 m_{0}$ for the NN case.  

When a strong magnetic field is applied, for example $B = 100$ T:  

\begin{itemize}
	\item[a)] Nearest neighbor (NN)
	\begin{itemize}
		\item At valley $K$: $m_{h} \approx 0.7011 m_{0},\; m_{e} \approx 0.4763 m_{0}$.  
		The reduced mass is $m_{r} \approx 0.2836 m_{0}$, which increases by $\approx 6.7\%$. 
		
		\item At valley $K'$: $m_{h} \approx 0.6597 m_{0},\; m_{e} \approx 0.4606 m_{0}$.  
		The reduced mass is $m_{r} \approx 0.2713 m_{0}$, which increases by $\approx 2.0\%$. 
	\end{itemize}
	\item[b)] Third nearest neighbor (TNN)
	\begin{itemize}
		\item At valley $K$: $m_{h} \approx 0.5739 m_{0},\; m_{e} \approx 0.4573 m_{0}$.  
		The reduced mass is $m_{r} \approx 0.2545 m_{0}$, which increases by $\approx 8.71\%$. 
		
		\item At valley $K'$: $m_{h} \approx 0.5584 m_{0},\; m_{e} \approx 0.4263 m_{0}$.  
		The reduced mass is $m_{r} \approx 0.2417 m_{0}$, which increases by $\approx 3.25\%$. 
	\end{itemize}
\end{itemize}

\begin{figure}[htb]
	\begin{subfigure}{0.495\textwidth}
		\centering
		\includegraphics[width=\linewidth]{imgs/MoS2/massK1.pdf}
	\end{subfigure}
	\begin{subfigure}{0.495\textwidth}
		\centering
		\includegraphics[width=\linewidth]{imgs/MoS2/massK2.pdf}
	\end{subfigure}
	\caption{Effective masses.}
\end{figure}

Meanwhile, Goryca \textit{et al.} \cite{goryca2019} reported that $m_{r} \approx 0.27 \pm 0.01 m_{0}$, which is $4\%-10.2\%$ larger than the earlier result of Berkelbach \textit{et al.} \cite{berkelbach2013}, $m_{r} = 0.245 \pm 0.005 m_{0}$.  
Based on our calculations, we argue that the reduced mass at valley $K$, $m_{r} \approx 0.2545 m_{0}$ with an increase of $8.71\%$, is consistent with the experimental findings of Goryca \textit{et al.}.


\newpage
\subsubsection*{Monolayer MoSe$_{2}$}
\begin{figure}[htb]
	\begin{subfigure}{0.495\textwidth}
		\centering
		\includegraphics[width=\linewidth]{imgs/MoSe2/ButAndWave_c.pdf}
	\end{subfigure}
	\begin{subfigure}{0.495\textwidth}
		\centering
		\includegraphics[width=\linewidth]{imgs/MoSe2/ButAndWave_v.pdf}
	\end{subfigure}
	\caption{Landau levels (a) and the corresponding envelope-function components (b),(c) for conduction electrons at valleys $K$ and $K'$. Figs (d)–(f) show the same as (a)–(c) but for valence electrons.}
\end{figure}
The band structure of MoSe$_{2}$ without a magnetic field shows that the $\Gamma$ point does not appear near the $K$ point. Therefore, when a magnetic field is applied, the $\Gamma$-point energy level is absent in this region. In addition, the first three Landau levels originate from the $K'$ valley, in contrast to WSe$_{2}$, where the first two Landau levels originate from the $K'$ valley.  

The effective masses of MoSe$_{2}$ in the absence of a magnetic field, calculated from
\[
\frac{1}{m_{ij}^{*}} = \frac{1}{\hbar^{2}} \frac{\partial^{2} E(\mathbf{k})}{\partial k_{i}\partial k_{j}},
\]
are $m_{e} \approx 0.4770 m_{0}$, $m_{h} \approx 0.5887 m_{0}$, and $m_{r} \approx 0.2634 m_{0}$ for the TNN case, and $m_{e} \approx 0.5226 m_{0}$, $m_{h} \approx 0.7512 m_{0}$, and $m_{r} \approx 0.3082 m_{0}$ for the NN case.  

When a strong magnetic field is applied, for example $B = 100$ T:  

\begin{itemize}
	\item[a)] Nearest neighbor (NN)
	\begin{itemize}
		\item At valley $K$: $m_{h} \approx 0.8100 m_{0},\; m_{e} \approx 0.5529 m_{0}$.  
		The reduced mass is $m_{r} \approx 0.3286 m_{0}$, which increases by $\approx 6.62\%$. 
		
		\item At valley $K'$: $m_{h} \approx 0.7632 m_{0},\; m_{e} \approx 0.5331 m_{0}$.  
		The reduced mass is $m_{r} \approx 0.3138 m_{0}$, which increases by $\approx 1.82\%$. 
	\end{itemize}
	\item[b)] Third nearest neighbor (TNN)
	\begin{itemize}
		\item At valley $K$: $m_{h} \approx 0.7168 m_{0},\; m_{e} \approx 0.5320 m_{0}$.  
		The reduced mass is $m_{r} \approx 0.3052 m_{0}$, which increases by $\approx 15.87\%$. 
		
		\item At valley $K'$: $m_{h} \approx 0.6251 m_{0},\; m_{e} \approx 0.4874 m_{0}$.  
		The reduced mass is $m_{r} \approx 0.2738 m_{0}$, which increases by $\approx 3.95\%$. 
	\end{itemize}
\end{itemize}

\begin{figure}[htb]
	\begin{subfigure}{0.495\textwidth}
		\centering
		\includegraphics[width=\linewidth]{imgs/MoSe2/massK1.pdf}
	\end{subfigure}
	\begin{subfigure}{0.495\textwidth}
		\centering
		\includegraphics[width=\linewidth]{imgs/MoSe2/massK2.pdf}
	\end{subfigure}
	\caption{Effective masses.}
\end{figure}

Meanwhile, Goryca \textit{et al.} \cite{goryca2019} reported that $m_{r} \approx 0.350 \pm 0.015 m_{0}$, which is $24.1\%-35.2\%$ larger than the earlier result of Berkelbach \textit{et al.} \cite{berkelbach2013}, $m_{r} = 0.27 m_{0}$.  
Based on our calculations, we argue that at valley $K$, the reduced mass $m_{r} \approx 0.3052 m_{0}$, with an increase of $15.87\%$, does not fully agree with the experimental findings of Goryca \textit{et al.}.






\newpage
\subsubsection*{Monolayer MoTe$_{2}$}
\begin{figure}[htb]
	\begin{subfigure}{0.495\textwidth}
		\centering
		\includegraphics[width=\linewidth]{imgs/MoTe2/ButAndWave_c.pdf}
	\end{subfigure}
	\begin{subfigure}{0.495\textwidth}
		\centering
		\includegraphics[width=\linewidth]{imgs/MoTe2/ButAndWave_v.pdf}
	\end{subfigure}
	\caption{Landau levels (a) and the corresponding envelope-function components (b),(c) for conduction electrons at valleys $K$ and $K'$. Figs (d)–(f) show the same as (a)–(c) but for valence electrons.}
\end{figure}
The band structure of MoTe$_{2}$ without a magnetic field shows that the $\Gamma$ point has an energy level of $E \approx -0.1075$ eV. Therefore, when a magnetic field is applied, this $\Gamma$-point energy level still appears.  

The effective masses of MoTe$_{2}$ in the absence of a magnetic field, calculated from  
\[
\frac{1}{m_{ij}^{*}} = \frac{1}{\hbar^{2}} \frac{\partial^{2} E(\mathbf{k})}{\partial k_{i}\partial k_{j}},
\]
are $m_{e} \approx 0.4318 m_{0}$, $m_{h} \approx 0.6044 m_{0}$, and $m_{r} \approx 0.2519 m_{0}$ for the TNN case, and $m_{e} \approx 0.5913 m_{0}$, $m_{h} \approx 0.8975 m_{0}$, and $m_{r} \approx 0.3565 m_{0}$ for the NN case, as also reported by Goryca \textit{et al.} \cite{goryca2019}. Among the six materials considered, MoTe$_{2}$ has the largest effective masses $m_{e}$ and $m_{h}$ in the absence of a magnetic field.  

When a strong magnetic field is applied, for example $B = 90$ T:  

\begin{itemize}
	\item[a)] Nearest neighbor (NN)
	\begin{itemize}
		\item At valley $K$: $m_{h} \approx 0.9774 m_{0},\; m_{e} \approx 0.6304 m_{0}$.  
		The reduced mass is $m_{r} \approx 0.3832 m_{0}$, which increases by $\approx 7.49\%$. 
		
		\item At valley $K'$: $m_{h} \approx 0.9142 m_{0},\; m_{e} \approx 0.6034 m_{0}$.  
		The reduced mass is $m_{r} \approx 0.3635 m_{0}$, which increases by $\approx 1.96\%$. 
	\end{itemize}
	\item[b)] Third nearest neighbor (TNN)
	\begin{itemize}
		\item At valley $K$: $m_{h} \approx 0.7704 m_{0},\; m_{e} \approx 0.4850 m_{0}$.  
		The reduced mass is $m_{r} \approx 0.2976 m_{0}$, which increases by $\approx 18.14\%$. 
		
		\item At valley $K'$: $m_{h} \approx 0.6322 m_{0},\; m_{e} \approx 0.4463 m_{0}$.  
		The reduced mass is $m_{r} \approx 0.2616 m_{0}$, which increases by $\approx 3.85\%$. 
	\end{itemize}
\end{itemize}

\begin{figure}[htb]
	\begin{subfigure}{0.495\textwidth}
		\centering
		\includegraphics[width=\linewidth]{imgs/MoTe2/massK1.pdf}
	\end{subfigure}
	\begin{subfigure}{0.495\textwidth}
		\centering
		\includegraphics[width=\linewidth]{imgs/MoTe2/massK2.pdf}
	\end{subfigure}
	\caption{Effective masses.}
\end{figure}

In the study of Goryca \textit{et al.} \cite{goryca2019}, the reduced mass was reported as $m_{r} = 0.36 \pm 0.04 m_{0}$, which is about 25\% larger than the value obtained in the work of Korm\'{a}nyos \textit{et al.} \cite{kormanyos2015}. In our case, for the TNN model, the reduced mass is $m_{r} = 0.2976 m_{0}$, which increases by $\approx 18\%$ compared to the zero-field value $m_{r} = 0.2519 m_{0}$.



\newpage
\subsubsection*{Monolayer WS$_{2}$}
\begin{figure}[htb]
	\begin{subfigure}{0.495\textwidth}
		\centering
		\includegraphics[width=\linewidth]{imgs/WS2/ButAndWave_c.pdf}
	\end{subfigure}
	\begin{subfigure}{0.495\textwidth}
		\centering
		\includegraphics[width=\linewidth]{imgs/WS2/ButAndWave_v.pdf}
	\end{subfigure}
	\caption{Landau levels (a) and the corresponding envelope-function components (b),(c) for conduction electrons at valleys $K$ and $K'$. Figs (d)–(f) show the same as (a)–(c) but for valence electrons.}
\end{figure}

The band structure of WS$_{2}$ without a magnetic field shows that the $\Gamma$ point has an energy level of $E \approx -0.1075$ (eV). Therefore, when a magnetic field is applied, the energy level at the $\Gamma$ point still appears.  

The effective mass of WS$_{2}$ without a magnetic field, calculated using
\[
\frac{1}{m_{ij}^{*}} = \frac{1}{\hbar^{2}} \frac{\partial^{2} E(\mathbf{k})}{\partial k_{i} k_{j}},
\]
yields $m_{e} \approx 0.2956 m_{0},\; m_{h} \approx 0.3845 m_{0},\; m_{r} \approx 0.1671 m_{0}$ in the TNN case, and $m_{e} \approx 0.3195 m_{0},\; m_{h} \approx 0.4348 m_{0},\; m_{r} \approx 0.1841 m_{0}$ in the NN case.  

For a strong magnetic field, e.g., $B = 100$ T:  
\begin{itemize}
	\item[a)] Nearest neighbor  
	\begin{itemize}
		\item At the K valley: $m_{h} \approx 0.4735 m_{0},\; m_{e} \approx 0.3389 m_{0}$. Thus, $m_{r} \approx 0.1974 m_{0}$, which increases by $\approx 7.2\%$.  
		\item At the K' valley: $m_{h} \approx 0.4438 m_{0},\; m_{e} \approx 0.3273 m_{0}$. Thus, $m_{r} \approx 0.1885 m_{0}$, which increases by $\approx 2.3\%$.  
	\end{itemize}
	\item[b)] Third nearest neighbor  
	\begin{itemize}
		\item At the K valley: $m_{h} \approx 0.4205 m_{0},\; m_{e} \approx 0.3275 m_{0}$. Thus, $m_{r} \approx 0.1841 m_{0}$, which increases by $\approx 10.17\%$.  
		\item At the K' valley: $m_{h} \approx 0.4043 m_{0},\; m_{e} \approx 0.3023 m_{0}$. Thus, $m_{r} \approx 0.1730 m_{0}$, which increases by $\approx 3.53\%$.  
	\end{itemize}
\end{itemize}

In the study of Goryca \textit{et al.} \cite{goryca2019}, they reported $m_{r} = 0.175 \pm 0.007 m_{0}$, which is about 10\% larger than the value $m_{r} = 0.15 - 0.16 m_{0}$ obtained in the work of Berkelbach \textit{et al.} \cite{berkelbach2013}.  

In our case, for the K valley under the TNN approximation, we obtain $m_{r} \approx 0.1841 m_{0}$ in the presence of a magnetic field, which corresponds to an increase of $\approx 10\%$ compared to $m_{r} \approx 0.1671 m_{0}$ without a magnetic field. This result is consistent with and reasonable compared to the experimental findings of Goryca \textit{et al.} \cite{goryca2019}.  


\begin{figure}[htb]
	\begin{subfigure}{0.495\textwidth}
		\centering
		\includegraphics[width=\linewidth]{imgs/WS2/massK1.pdf}
	\end{subfigure}
	\begin{subfigure}{0.495\textwidth}
		\centering
		\includegraphics[width=\linewidth]{imgs/WS2/massK2.pdf}
	\end{subfigure}
	\caption{Khối lượng hiệu dụng.}
\end{figure}
\newpage
\subsubsection*{Monolayer WSe$_{2}$}
\begin{figure}[htb]
	\begin{subfigure}{0.495\textwidth}
		\centering
		\includegraphics[width=\linewidth]{imgs/WSe2/ButAndWave_c.pdf}
	\end{subfigure}
	\begin{subfigure}{0.495\textwidth}
		\centering
		\includegraphics[width=\linewidth]{imgs/WSe2/ButAndWave_v.pdf}
	\end{subfigure}
	\caption{Landau levels (a) and the corresponding envelope-function components (b),(c) for conduction electrons at valleys $K$ and $K'$. Figs (d)–(f) show the same as (a)–(c) but for valence electrons.}
\end{figure}
The band structure of WSe$_{2}$ without a magnetic field shows that the $\Gamma$ point lies much lower in energy than the K point. Therefore, when a magnetic field is applied, the energy level at the $\Gamma$ point does not appear here.  

The effective mass of WSe$_{2}$ without a magnetic field, calculated using
\[
\frac{1}{m_{ij}^{*}} = \frac{1}{\hbar^{2}} \frac{\partial^{2} E(\mathbf{k})}{\partial k_{i} k_{j}},
\]
yields $m_{e} \approx 0.3124 m_{0},\; m_{h} \approx 0.4022 m_{0},\; m_{r} \approx 0.1758 m_{0}$ in the TNN case, and $m_{e} \approx 0.3487 m_{0},\; m_{h} \approx 0.4792 m_{0},\; m_{r} \approx 0.2018 m_{0}$ in the NN case, as reported in the works of Kyl\"{a}np\"{a}\"{a} \textit{et al.} and Berkelbach \textit{et al.} \cite{kylanpaa2015,berkelbach2013}.  

For a strong magnetic field, e.g., $B = 100$ T:  
\begin{itemize}
	\item[a)] Nearest neighbor  
	\begin{itemize}
		\item At the K valley: $m_{h} \approx 0.5220 m_{0},\; m_{e} \approx 0.3702 m_{0}$. Thus, $m_{r} \approx 0.2166 m_{0}$, which increases by $\approx 7.34\%$.  
		\item At the K' valley: $m_{h} \approx 0.4888 m_{0},\; m_{e} \approx 0.3573 m_{0}$. Thus, $m_{r} \approx 0.2064 m_{0}$, which increases by $\approx 2.28\%$.  
	\end{itemize}
	\item[b)] Third nearest neighbor  
	\begin{itemize}
		\item At the K valley: $m_{h} \approx 0.4417 m_{0},\; m_{e} \approx 0.3494 m_{0}$. Thus, $m_{r} \approx 0.1951 m_{0}$, which increases by $\approx 10.98\%$.  
		\item At the K' valley: $m_{h} \approx 0.4257 m_{0},\; m_{e} \approx 0.3199 m_{0}$. Thus, $m_{r} \approx 0.1826 m_{0}$, which increases by $\approx 3.87\%$.  
	\end{itemize}
\end{itemize}

In the study of Stier \textit{et al.} \cite{stier2018}, they reported $m_{r} \approx 0.20 \pm 0.01 m_{0}$, which is about 15\% larger than the predictions of recent theoretical works \cite{berkelbach2013,kylanpaa2015}. In our case, for the K valley under the TNN approximation, we obtain $m_{r} \approx 0.1951 m_{0}$ in the presence of a magnetic field, corresponding to an increase of $\approx 11\%$ compared to the value without a magnetic field. This result is consistent with the findings reported by Stier \textit{et al.} \cite{stier2018}.  

\begin{figure}[htb]
	\begin{subfigure}{0.495\textwidth}
		\centering
		\includegraphics[width=\linewidth]{imgs/WSe2/massK1.pdf}
	\end{subfigure}
	\begin{subfigure}{0.495\textwidth}
		\centering
		\includegraphics[width=\linewidth]{imgs/WSe2/massK2.pdf}
	\end{subfigure}
	\caption{Khối lượng hiệu dụng.}
\end{figure}


\newpage
\subsubsection*{Monolayer WTe$_{2}$}
\begin{figure}[htb]
	\begin{subfigure}{0.495\textwidth}
		\centering
		\includegraphics[width=\linewidth]{imgs/WTe2/ButAndWave_c.pdf}
	\end{subfigure}
	\begin{subfigure}{0.495\textwidth}
		\centering
		\includegraphics[width=\linewidth]{imgs/WTe2/ButAndWave_v.pdf}
	\end{subfigure}
	\caption{Landau levels (a) and the corresponding envelope-function components (b),(c) for conduction electrons at valleys $K$ and $K'$. Figs (d)–(f) show the same as (a)–(c) but for valence electrons.}
\end{figure}

The band structure of WTe$_{2}$ in the absence of a magnetic field shows that the $\Gamma$ point lies significantly lower in energy than the K point. Therefore, under an applied magnetic field, the Landau levels associated with the $\Gamma$ point do not appear in this regime.  

The effective masses of WTe$_{2}$ without a magnetic field, calculated using the formula $\frac{1}{m_{ij}^*} = \frac{1}{\hbar^{2}} \frac{\partial^{2}E(\mathbf{k})}{\partial k_{i} k_{j}}$, are found to be $m_{e} \approx 0.2478 m_{0}$, $m_{h} \approx 0.3332 m_{0}$, and $m_{r} \approx 0.1421 m_{0}$ for the TNN case, and $m_{e} \approx 0.3169 m_{0}$, $m_{h} \approx 0.4559 m_{0}$, and $m_{r} \approx 0.187 m_{0}$ for the NN case. To the best of our knowledge, no previous studies have reported the effective masses of WTe$_{2}$.  

At a high magnetic field, for example $B = 80$ T, the effective masses are obtained as follows:  

\begin{itemize}
	\item[a)] Nearest neighbor
	\begin{itemize}
		\item At the K valley: $m_{h} \approx 0.5093 m_{0},\; m_{e} \approx 0.3413 m_{0}$. \\
		This yields $m_{r} \approx 0.2044 m_{0}$, corresponding to an increase of $\approx 9.3\%$.  
		
		\item At the K$'$ valley: $m_{h} \approx 0.4681 m_{0},\; m_{e} \approx 0.3264 m_{0}$. \\
		This gives $m_{r} \approx 0.1923 m_{0}$, corresponding to an increase of $\approx 2.83\%$.  
	\end{itemize}
	\item[b)] Third nearest neighbor
	\begin{itemize}
		\item At the K valley: $m_{h} \approx 0.387 m_{0},\; m_{e} \approx 0.2824 m_{0}$. \\
		This yields $m_{r} \approx 0.1633 m_{0}$, corresponding to an increase of $\approx 14.92\%$.  
		
		\item At the K$'$ valley: $m_{h} \approx 0.3562 m_{0},\; m_{e} \approx 0.2577 m_{0}$. \\
		This gives $m_{r} \approx 0.1495 m_{0}$, corresponding to an increase of $\approx 5.21\%$.  
	\end{itemize}
\end{itemize}

Thus, at the K valley in the TNN case, the reduced mass $m_r$ of WTe$_{2}$ exhibits an increasing of nearly $15\%$ under a strong magnetic field, which is consistent with the trend observed for other materials discussed above.  

\begin{figure}[htb]
	\begin{subfigure}{0.495\textwidth}
		\centering
		\includegraphics[width=\linewidth]{imgs/WTe2/massK1.pdf}
	\end{subfigure}
	\begin{subfigure}{0.495\textwidth}
		\centering
		\includegraphics[width=\linewidth]{imgs/WTe2/massK2.pdf}
	\end{subfigure}
	\caption{Khối lượng hiệu dụng.}
\end{figure}

\newpage

\bibliographystyle{unsrt}
\bibliography{refs}

\end{document}
